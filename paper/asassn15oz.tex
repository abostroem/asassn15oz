% mnras_template.tex
%
% LaTeX template for creating an MNRAS paper
%
% v3.0 released 14 May 2015
% (version numbers match those of mnras.cls)
%
% Copyright (C) Royal Astronomical Society 2015
% Authors:
% Keith T. Smith (Royal Astronomical Society)

% Change log
%
% v3.0 May 2015
%    Renamed to match the new package name
%    Version number matches mnras.cls
%    A few minor tweaks to wording
% v1.0 September 2013
%    Beta testing only - never publicly released
%    First version: a simple (ish) template for creating an MNRAS paper

%%%%%%%%%%%%%%%%%%%%%%%%%%%%%%%%%%%%%%%%%%%%%%%%%%
% Basic setup. Most papers should leave these options alone.
\documentclass[a4paper,fleqn,usenatbib]{mnras}

% MNRAS is set in Times font. If you don't have this installed (most LaTeX
% installations will be fine) or prefer the old Computer Modern fonts, comment
% out the following line
\usepackage{newtxtext,newtxmath}
% Depending on your LaTeX fonts installation, you might get better results with one of these:
%\usepackage{mathptmx}
%\usepackage{txfonts}

% Use vector fonts, so it zooms properly in on-screen viewing software
% Don't change these lines unless you know what you are doing
\usepackage[T1]{fontenc}
\usepackage{ae,aecompl}


%%%%% AUTHORS - PLACE YOUR OWN PACKAGES HERE %%%%%
\usepackage{geometry}                		% See geometry.pdf to learn the layout options. There are lots.
\geometry{letterpaper}                   		% ... or a4paper or a5paper or ... 
\usepackage{graphicx}				% Use pdf, png, jpg, or eps§ with pdflatex; use eps in DVI mode
								% TeX will automatically convert eps --> pdf in pdflatex	
\graphicspath{ {figures/} }	
\usepackage{amssymb}

\usepackage{natbib}
\usepackage{color} 
\usepackage{deluxetable}
\usepackage{cancel}

\makeatletter
\def\@to{to}
\makeatother

%%%%%%%%%%%%%%%%%%%%%%%%%%%%%%%%%%%%%%%%%%%%%%%%%%

%%%%% AUTHORS - PLACE YOUR OWN COMMANDS HERE %%%%%

\newcommand{\azaleecomment}[1]{{\color{red} [{#1}]}} 
\newcommand{\Azalee}[1]{\azaleecomment{Azalee: #1}}
\newcommand{\viktoriyacomment}[1]{{\color{blue} [{#1}]}} 
\newcommand{\Viktoriya}[1]{\viktoriyacomment{Viktoriya: #1}}
\newcommand{\stefanocomment}[1]{{\color{cyan} [{#1}]}} 
\newcommand{\Stefano}[1]{\stefanocomment{Stefano: #1}}


\newcommand{\msun}{M$_{\odot}$ }
\newcommand{\msunperiod}{M$_{\odot}$}
\newcommand{\lsun}{L$_{\odot}$}
\newcommand{\rsun}{R$_{\odot}$}

%%%%%%%%%%%%%%%%%%% TITLE PAGE %%%%%%%%%%%%%%%%%%%

% Title of the paper, and the short title which is used in the headers.
% Keep the title short and informative.
\title{ASASSN-15oz}

% The list of authors, and the short list which is used in the headers.
% If you need two or more lines of authors, add an extra line using \newauthor

\author[K. A. Bostroem et al.]{K. Azalee Bostroem, $^{1}$\thanks{E-mail: kabostroem@ucdavis.edu}
Stefano Valenti,$^{1}$
Assaf Horesh,$^{2}$
Viktoriya Morozova,$^{3}$
\newauthor 
Paul Kuin,$^{4}$
Samuel Wyatt,$^{5}$
Anders Jerkstrand,$^{6}$
David Sand,$^{5}$
\\
% List of institutions
$^{1}$University of California, Davis, 1 Shields Ave, Davis, CA, 95616 \\
$^{2}$Hebrew University of Jerusalem \\
$^{3}$Princeton University \\
$^{4}$Mullard Space Science Laboratory - University College London\ \
$^{5}$University of Arizona \\
$^{6}$ Max-Planck Institut f�r Astrophysik, Karl-Schwarzschild Str. 1, D-85748 Garching, Germany
}

% These dates will be filled out by the publisher
\date{Accepted XXX. Received YYY; in original form ZZZ}

% Enter the current year, for the copyright statements etc.
\pubyear{2018}

% Don't change these lines
\begin{document}
\label{firstpage}
\pagerange{\pageref{firstpage}--\pageref{lastpage}}
\maketitle

% Abstract of the paper
\begin{abstract}

Single, hydrogen-rich, core collapse supernovae typically divided into three classes: IIP, with a 100 day plateau following maximum light, IIL, showing a linearly declining light curve following maximum light, and IIn which show narrow emission lines in their spectra. These classification are indicative of the characteristics of the late stage evolution of the hydrogen envelope of the progenitor star. The plateau is explained by recombination through a thick envelope, the linear decline with recombination through a thinner envelope, and IIn with strong interaction with circumstellar material lost prior to explosion. For this reason, we generally do not consider interaction when modeling IIP and IIL supernovae. However, recent hydrodynamic modeling of IIP and IIL supernovae has been unable to reproduce the early (~30 days post explosion) light curve without the introduction of circumstellar material. One explanation for the smaller hydrogen envelope in IIL supernovae is mass loss, which would lead to circumstellar material. This makes IIL supernovae ideal for understanding whether IIP and IIL supernovae experience interaction. We test this hypothesis on ASASSN-15oz, a type IIL supernova and the closest type II supernova in 2015. With extensive follow-up in the x-ray, UV, optical, IR, and radio we search for signs of interaction. Interestingly, we find evidence of interaction in the radio and optical, but only upper limits in the X-ray and UV, leading to questions about universality of these indicators.
\end{abstract}
% Select between one and six entries from the list of approved keywords.
% Don't make up new ones.
\begin{keywords}
Supernovae
\end{keywords}

%%%%%%%%%%%%%%%%%%%%%%%%%%%%%%%%%%%%%%%%%%%%%%%%%%

%%%%%%%%%%%%%%%%% BODY OF PAPER %%%%%%%%%%%%%%%%%%
%%%%%%%%%%%%%%%%% INTRODUCTION %%%%%%%%%%%%%%%%%%
\clearpage
\begin{itemize}
\item \Azalee{Cite Instruments}
\item \Azalee{Variations in level of detail in the descriptions of data reduction}
\end{itemize}
\section{Introduction}
({\color{blue} I would start directly with massive star hydrogen rich SNe II) instead of CC SNe in general (you do not speak about stripped envelope right?)} 
Stars greater than 8 \msun end their lives as spectacularly bright core-collapse supernovae (CCSN). 
With the advent of wide field survies, it is increasingly easier to observe the supernova phase, it is challenging more to connect the supernova observations to their progenitor star. {\color{blue} (It seems that the challenge in connecting is related to wide fields surveys. what about: Despite an increasing amount of high quality data on SNe II, we are still struggling to connect the SN observation ... )}
This is due in part to uncertainties in the late stages of the evolution of massive stars and in part to the challenges of modeling the complex physics of the supernova explosion and the expansion of the ejecta.

\Azalee{Figure out how to weave this into the next paragraph: This interaction should be easiest to detect in type IIL supernovae as they have more of their hydrogen envelope than IIP supernovae.}
One of the biggest challenges in connecting supernova observations to their progenitors is understanding the mass-loss of massive stars.
The supernova that show hydrogen in their spectra throughout their evolution (type II) are currently attributed to single star progenitors (need citation).
These supernova generally increase in brightness for a week and then decline linearly for $\sim$80 days, a period referred to as the photospheric phase, before dropping dramatically onto another linear decline, called the nebular phase (cite: Barbon, 1979, ). 
%\Stefano{include a phase of decline between max and plateau?} {\color{blue} (not needed)}.
However, within this general shape there is a great diversity in rise times, length of the photospheric phase, height of the drop to the nebular phase, and slopes of each decline (see for example Bertaud, 1941 (103), Minkowski 1964, Anderson, 2014, Valenti 2017). 
In fact, the there are two classes of type II supernova, defined based on the light curve shape during the photospheric phase.
Those that stay at constant brightness, creating a plateau, are called type IIP supernovae, while those that decline linearly are referred to at type IIL supernovae (cite: Barbon, 1979). 
It is commonly accepted that the difference in light curve shape is due to the amount of hydrogen in the outer envelope: IIL supernovae having a thinner envelope (Find citation). 
The most natural mechanism for this is through mass-loss due to stellar winds (find citation). 
This view is supported by the apparent continuum of light curves found between the two classes (see for example Anderson 2014).
However, until recently IIL and IIP supernovae were modeled without CSM interaction, an assumption at odds with the idea of late stage mass-loss (why? citations?).

New full radiative transfer hydrodynamic light curve modeling of type IIP/IIL supernovae have required some amount of CSM to reproduce the slope of the light curve during the first 30-40 days (cite Viktoriya and MESA/STELLA paper) for most supernovae.
Additionally, narrow emission lines in the spectra of some supernovae taken a few hours to days after explosion provide more evidence of interaction with CSM (citation: nature paper - Gal Yam?). 
However, a comprehensive search for signs of interaction at all wavelengths, including those most commonly associated with interaction, X-ray and Radio, has not been done. 
In this paper, for the first time we address this issue with the study of ASASSN-15oz, a type IIL supernova, for which we were able to collect and extensive set of data ranging from the X-ray to the Radio to search for signs of interaction with a CSM.

The outline of the paper is as follows.
In Section \ref{15ozIntro} we present the explosion parameters of ASASSN-15oz.
In Section \ref{sec:Obs} we present the data collected at all wavelengths.
The photometric and spectroscopic evolution are described in Sections \ref{sec:LCEvolve} and \ref{sec:SpecEvolve}, respectively.
Section \ref{sec:Interaction} describes our search for interaction, including extensive hydrodynamic light curve modeling.
Finally, we present a comparison to other objects and conclusions in \ref{SecComp}.
\Stefano{should I say something about if there's no CSM?}

%%%%%%%%%%%%%%%%% SN PARAMETERS %%%%%%%%%%%%%%%%%%
\section{Supernova Parameters} \label{15ozIntro}
Supernova ASASSN-15oz was discovered by the ASAS-SN team on August 31, 2015 and announced on September 3, 2015 \citep{2016brown}. 
The next day Las Cumbres Observatory (LCO) classified it as a type II supernova one week after explosion \citep{2016hosseinzadeh}. 
We use a distance modulus to the host galaxy, HIPASS J1919-33 we use the Hubble Flow Distance Modulus $\mu=32.3\pm 0.02$ with H$_{0}$=73 km/s/Mpc corrected for Virgo infall.
\Stefano{cite NED?}

Imaging of HIPASS J1919-33 from August 23, 2015 does not show any evidence of ASASSN-15oz (V $>$ 17.8). 
We take this to be a lower limit on the explosion epoch. 
It is therefore reasonable to take as the mid-point between the non-detection and the detection as explosion date with the $1\sigma$ error being the time to the non-detection: $t_{expl} =$ 2015-08-27 $\pm 4$ days.

\citet{2017gutierrez} found that the explosion epoch can be determined, with an average errors of 4.9 days, by fitting the blue part of the first spectrum with SNID \citep{2011blondin} and using the explosion epoch of the best fit template.
This method relies on the existence of a template supernova that is similar in both phase and type to the object being fit.
ASASSN-15oz is a type IIL supernova and our first spectrum is from around maximum. 
SNID has very few templates for type IIL supernovae, we therefore add templates for two well observed supernovae with well constrained explosion epochs: 2012A and 2013ej.
We confirm our explosion epoch by fitting the first spectrum (2015-09-04) using SNID v. 5.0, augmenting the templates with those of \citet{2017gutierrez}, SN2012A \citep{2013tomasella}, and SN2013ej \citep{2016childress,2016dhungana,2014valenti,2015smartt}.
The five best fit spectra imply an explosion epoch of  2015-08-22, consistent with the range we propose based on non-detections.
interestingly, the hydrodynamic light curve modeling also places the best fit explosion epoch on the day of the last non-detection (see Section \ref{sec:LCEvolve}).

We use an X-shooter spectrum taken with the UVB and VIS arms to constrain the host galaxy reddening to confirm the galactic reddening using the equivalent widths of the Na I D1 and D2 lines ($\lambda$5890, $\lambda$5896) \citep{2012poznanski}. 
We find no evidence of Na I absorption at the host redshift and therefore assume no host absorption. 
By fitting gaussian profiles to the galactic Na I absorption we find E(B-V) = 0.12, consistent with the value given by \citet{2011schlafly}. 
We adopt the \citet{2011schlafly}  value of 0.08 for the analysis in this paper. 
%%%%%%%%%%%%%%%%% OBSERVATIONS and DATA REDUCTION %%%%%%%%%%%%%%%%%%
\section{Observations and Data Reduction}  \label{sec:Obs}
%\Azalee{Look into how people talk about telescopes vs collaborations vs instruments}
ASASSN-15oz was the closest type II supernova of 2015 and was observed extensively at all wavelengths. 
The majority of our observations were provided by the Las Cumbres Observatory (LCO) Supernova Key Project (2014-2017) and the Public ESO Spectroscopic Survey for Transients (PESSTO). 
We supplemented these observations with dedicated projects in the X-ray and UV with \textit{Swift} (1114241, PI: Valenti) and in the radio with the VLA (VLA/15B-362, PI: Valenti). 
The photometric and spectroscopic data reduction are defined in the next sections. 
A complete list of photometric observations is given in Table \ref{tab:LcObs}.  
Spectroscopic observations are detailed in Table \ref{tab:SpecObs}.
%%%%%%%%%%%%%%%%% 
\subsection{Optical}
The optical light curve was closely monitored by the Las Cumbres Observatory (LCO) from discovery through day 400 in the filters B, V, g, r, and i. 
Unfortunately, the supernova passed behind the sun around day 80, leaving a gap in the observations from day 86 to day 184, most notably missing the fall from plateau.
Figure \ref{fig:LC} shows the complete light curve.
LCO observations were reduced with the LCO imaging pipeline, lcogtsnpipe \citep{2016valenti}. 
This pipeline employs PSF photometry, removing background contamination by fitting a low order polynomial to the host galaxy.
Instrument magnitudes are converted to absolute magnitudes using the APASS system \footnote{https://www.aavso.org/apass} converting B and V to the Landolt magnitude system (Vega magnitudes) and g, r, and i to the sloan magnitude system (AB magnitudes). 
%=======================
\begin{figure}
\begin{center}
\includegraphics[width=\columnwidth]{full_lc}
\caption{The complete optical and UV light curve of ASASSN-15oz expressed as apparent magnitude (+offset) vs days since explosion. 
Each color corresponds to a single filter, labeled in the legend. 
The optical Swift filters are denoted with an "s" in the name (e.g. us).
Each filter has been shifted by a global offset (denoted in the legend) for viewing purposes.}
\label{fig:LC}
\end{center}
\end{figure}
%-----------------------------------------
Extensive optical spectroscopy was obtained during the photospheric phase with the FLOYDS spectrograph on the 2m LCO telescope at Siding Springs, Australia as part of the LCO Supernova Key Project and the EFOSC2 spectrograph on the 3.6 m NTT telescope at La Silla, Chile as part of the PESSTO Project. 
The PESSTO project also contributed late time nebular spectra that were observed with EFOSC2. 
One-dimensional spectra were extracted and calibrated using the FLOYDS \citep{2014valenti} and EFOSC tasks in the PESSTO pipeline \citep{2015smartt}. 
Both of these pipelines combine IRAF tasks to bias subtract and flat field the data and locate, extract, wavelength calibrate, and flux calibrate the spectrum.
\Stefano{Should I put (maybe in an appendix) which standard I used for which night?}
Figure \ref{fig:SpecAll} shows the evolution of the spectra over time, starting from a blue spectrum with broad hydrogen lines and developing metal features such as iron and scandium as the velocities decrease and the line widths narrow. 
%More unusually, the cachito feature \citep{2017gutierrez}, blueward of the H-$\alpha$ absorption is present in the first spectrum, growing in strength through day 20 (2015-09-16) and then decreasing in strength at later times. \Azalee{Say more here or reference future section for interpretation} \Stefano{Is this too much interpretation for this section?}
%=======================
\begin{figure}
\begin{center}
\includegraphics[width=\columnwidth]{spectra_montage}
\caption{The photospheric spectral evolution of ASASSN-15oz with the earliest spectrum on the bottom.
The first spectrum taken near maximum light shows hydrogen features (identified at the bottom of the figure).
At this early phase, the cachito feature (see Section \ref{sec:cachito}) is already visible. 
Over time the cachito feature fades while the hydrogen emission grows and metal lines become visible and grow in strength. 
These lines are identified in the top spectrum. }
\label{fig:SpecAll}
\end{center}
\end{figure}
%-----------------------------------------
%%%%%%%%%%%%%%%%% 
\subsection{UV and X-Ray Observations}\label{SecSwift}
UV and X-Ray imaging observations were obtained concurrently with \textit{Swift} programs 1114241 and XXXXXXX from 2015-09-05 through 2015-11-18.
In addition to the imaging observations, two epochs of \textit{Swift} grism observations were taken under program 1114241. 
Imaging observations were processed with aperture photometry following \citet{2009brown} employing the updated zeropoints of \citet{2010breeveld}.
UV grism observations were taken at two roll angles and extracted using the UVOTPY pipeline \citep{2014kuin}. 
There is a nearby star coincident with the supernova at one roll angle and just above it at another roll angle.
Although we tried custom extractions using different extraction box height and template subtraction using TRUVOT \citep{2015smitka} we were unable to produce a better calibration than the default pipeline. 
Due to the distance of the ASASSN-15oz, only the first of the two spectroscopic epochs has sufficient S/N to be extracted. 

An upper limit in the X-ray was determined using aperture photometry .
An 18" aperture was used to integrate around the location of ASASSN-15oz. 
The background was removed using an aperture in a blank part of the image and measured fluxes were converted from counts to luminosity assuming a power-law model with a photon index of two.
%%%%%%%%%%%%%%%%% 
\subsection{NIR Observations}
ASASSN-15oz was observed in the near infra-red (NIR) on September 05, 2015 and October 05, 2015 with the SOFI instrument and on October 10, 2015 with the SpeX instrument  \citep{2003rayner}.
Reduced spectra are shown in Figure \ref{fig:IRmontage}

The SpeX instrument is part of the 3.0-m NASA Infrared Telescope Facility (IRTF).  
The data were taken in so-called SXD mode, where the spectrum is cross-dispersed to obtain wavelength coverage from $\sim$0.8--2.4 $\mu$m in a single exposure, spread over six orders.  
All observations were taken with the slit aligned along the parallactic angle, and we employed a classic ABBA technique for improved sky subtraction.  
An A0V star was observed adjacent in time to the science observations for flux and telluric calibration.  
The spectrum was reduced in a standard way using the publicly available {\sc Spextool} software package \citep{2004cushing} and corrections for telluric absorption utilized {\sc XTELLCOR} \citep{2003vacca} and the A0V star observations.

The SOFI instrument is mounted on the NTT telescope and observations were obtained through the PESSTO collaboration. 
These spectra, taken on (day 8 and 38, respectively), were reduced with the SOFI pipeline in the PESSTO package \citep{2015smartt}.
%%%%%%%%%%%%%%%%% 
\subsection{Radio Observations}
We observed ASASSN\,15oz using the Jansky Very Large Array (VLA) on 2015 September 24 and October 08. In both of the observations we used J1924-2914 and 3C48 as a phase calibrator and a flux calibrator, respectively. We used the Common Astronomy Software Applications (CASA) standard packages and pipelines to perform the data calibration and imaging. The SN was detected in both observation with the following flux measurements: September 24 - $120\pm 23\,\mu$Jy at $4.8$\,GHz (C-band) and a $3\,\sigma$ non null detection limit at $22$\,GHz (K-band); October 08 -  $210\pm 21\,\mu$Jy at $4.8$\,GHz (C-band) and a $80\pm 17\,\mu$Jy at $15$\,GHz (KU-band). 
%%%%%%%%%%%%%%%%% LIGHT CURVE EVOLUTION %%%%%%%%%%%%%%%%%%
\section{Light Curve Evolution} \label{sec:LCEvolve}
The light curve of all type IIP/IIL supernovae show an initial rise, followed by a period of 100 days during which the supernova stays the same brightness or declines linearly. 
During this time the light curve is powered by the radioactive decay of ${}^{56}$Ni and the expansion of the ejecta approximately balances the receding Hydrogen recombination front, leading to a close to constant photospheric radius. 
About 100 days after explosion, the recombination front passes the inner edge of the hydrogen envelope causing a sudden drop in luminosity referred to as the fall from plateau.  
After the fall from plateau, light curve declines linearly, following the ${}^{56}$Co decay.

The light curve of ASASSN-15oz shows a linear decline similar to that of other type IIL SNe. 
The light curve reaches peak brightness in the V-band 8.25 days after explosion, one day after observations began, reaching an above average absolute magnitude of -18.05 $\pm$ 0.025.
The early light curve rises until around day 20 in the redder filters, flattening in the V-band and falling in the blue bands during this same period. 
The light curve then falls in all bands for the remainder of the observations, changing slope around day 40 and again sometime after day 85, following the fall from plateau. 
Following \citep{2014Anderson}, we fit a slope to the first steep initial decline after maximum ($s_1$) and to the second shallower slope prior to the fall from plateau ($s_2$). 
We also fit a global decline rate, $s_{50v}$ from soon after maximum to prior to the fall from plateau.
During the nebular phase we fit a slope to the radioactive decay tail ($s_{tail}$). 
The slopes we fit as well as the details of the fit are summarized in table \ref{tab:slope} and compared with other objects in Section \ref{SecComp}.
\Azalee{significant figures?}
%=======================
\begin{table*}
\centering
\caption{The best-fit slope to the V-band light curve of ASASSN-15oz measured between the start and end phase listed in the table. All slopes are measured in units of magnitudes per 50 days.}
\begin{tabular}{ccccc}
Slope Type & Slope (mag/50 days) & $\sigma_{slope}$ & Start Phase & End Phase \\
\hline
$s_1$ & 1.21 & 0.066 & 21.0 & 38.0 \\
$s_2$ & 0.99 & 0.021 & 45.0 & 78.0 \\
$s_{50V}$ & 1.09 & 0.014 & 18.4 & 78.0 \\
$s_{tail}$ & 0.70 & 0.029 & 206.9 & 368.0 \\
\end{tabular}
\end{table*}
  \label{tab:slope}%created by lightcurve_characteristics.ipynb
%-----------------------------------------
The slope of the radioactive decay tail is steeper than a pure ${}^{56}Co$ decay implying that the hydrogen envelop is not fully trapping and reprocessing the gamma rays produced by the radioactive decay.

The Pseudo-bolometric luminosity is computed following \citet{2008valenti}.
Each apparent magnitude is corrected for galactic extinction (see Section \ref{15ozIntro}).
The flux density is calculated from the apparent magnitude at the effective wavelength of the filter and integrated using Simpson's Rule. 
Finally, the flux is converted to a luminosity using the distance modulus (Section \ref{15ozIntro}). 
We use  the U, B, g, V, r, R, i, I to calculate the bolometric luminosity at the V-band cadence.
When V-band observations extended either earlier or later than a filter, we use the color with the neighboring filter to extrapolate the flux.
%%%%%%%%%%%%%%%%% SPECTROSCOPIC EVOLUTION%%%%%%%%%%%%%%%%%%
\section{Spectroscopic Evolution} \label{sec:SpecEvolve}
\Stefano{Should I have citations in this intro? Am I promising too much analysis that I'm not doing?}
The spectroscopic evolution of a SNe during the photospheric phase provides insight into details of the SN ejecta.
The changes in velocity and the individual line profiles describe the geometry and energetics of the ejecta. 
The presence of different species at different times is gives information about the chemical composition, temperature, and density of the ejecta.
In this section we analyze optical and NIR spectra taken throughout the photospheric phase.
%%%%%%%%%%%%%%%%
\subsection{Optical Evolution}
Initial line identification was performed by comparing the spectra to that of supernova 1999em \citep{2001leonard}. 
Identification was confirmed using the spectrum synthesis code Syn++ \citep{2013thomas} (parameter details in \ref{tab:syn++}. 
%=======================
\begin{table*}
\caption{The parameters used in the Syn++ fit. 
The opacity $\tau$ is modeled with an exponential profile with e-folding length aux, minimum velocity $v_{min}$, and maximum velocity $v_{max}$. 
The temperature column is the Boltzmann excitation temperature.}
\begin{center}
\begin{tabular}{c|c|c|c|c|c}
Ion & log($\tau$) & $v_{min}(kkm/s)$ & $v_{max}(kkm/s)$ & aux & Temperature (kK) \\
\hline
  H I -$\alpha$   & .06 & 0.1 & 40.0 & 2.0 & 10.0 \\
  H I - $\beta$   & .06 & 0.1 & 40.0 & 2.0 & 10.0 \\
  Na II & -0.5 & 0.1 & 40.0 & 1.0 & 10.0 \\
  O I & -0.8 & 0.1 & 40.0 & 1.0 & 10.0 \\
  Ca II & 1.3 & 0.1 & 40.0 & 2.0 & 10.0 \\
  Sc II & 0.1 & 0.1 & 40.0 & 1.0 & 10.0 \\
  Ti II  & 0.3 & 0.1 & 40.0 & 1.0 & 10.0 \\
  Fe I  &  0.3 & 0.1 & 40.0 & 1.0 & 10.0 \\
  Fe II & 0.3 & 0.1 & 40.0 & 1.0 & 7.0 \\
  Ba I & 0.0 & 0.1 & 40.0 & 1.0 & 10.0 \\  
 \end{tabular}
\end{center}
\label{tab:syn++}
\end{table*}
%-----------------------------------------
The results of the Syn++ fit can be seen in Figure \ref{fig:syn++}.
The top plot shows the combined fit, while the contributions from the individual elements are shown in the subsequent plots below.
Syn++ models pure resonance scattering which is a good approximation for all lines modeled except H-$\alpha$. 
For this reason we use the H-$\beta$ line to determine the hydrogen contribution to the fit. 
A separate fit to H-$\alpha$ is shown in the second plot of Figure \ref{fig:syn++} and the parameters are detailed in the H-$\alpha$ row of Table \ref{tab:syn++}.
%=======================
\begin{figure*}
\begin{center}
\includegraphics[height=\textwidth]{syn++indiv_elements} %created by make_syn++_indiv_element_fig.ipynb
\caption{A comparison of the model spectrum produced by Syn++ and the observed spectrum from 2015-10-06.
The top panel shows the best fit with all elements while the remaining panels show the fit for each element individually.
We fit H-$\alpha$ and H-$\beta$ separately as H-$\alpha$ is not well modeled by the pure resonance scattering assumed in Syn++.
The H-$\beta$ fit is used in the combined spectrum in the top panel.}
\label{fig:syn++}
\end{center}
\end{figure*}
%-----------------------------------------

We find the velocity of lines for which individual components can be resolved using two methods with the goal of making the velocity measurements robust to small deviations in the input parameters. 
For lines that are not blended or that are dominated by the absorption of one ion (Scandium features ($\lambda$ 5526, 5662, 6262)), we employ the method presented in \citet{2012silverman}. 
Briefly, the continuum is a straight line through the edges of a blended feature, defined by a concave down quadratic fit to the region where the slope changed. 
The velocity is then defined as the minimum of cubic spline fit  to the continuum subtracted flux (see \ref{AppLineFit} for more details).
Rather than expressing an uncertainty on the fit, we calculate the uncertainty in the velocity as the range of velocities that encompass 68.2\% of the integrated (continuum subtracted) flux around the feature minimum.
Since each single line is formed in an extended region of the ejecta, this represents the range of velocities at which each line forms.

For lines that are blended (but still showed isolated absorption troughs for some features) we define the continuum as a straight line from the lowest side of the feature. 
After dividing by the continuum, we then simultaneously fit multiple 1D Gaussian profiles to the continuum normalized spectra. 
For blended features originating in the same part of the ejecta, we require that each component has the same width.
We define the velocity of a feature as the minimum of the individual Gaussian profile corresponding to that ion and the error as the standard deviation of the fit. 
For the Ca II NIR triplet we use the additional constraint of a fixed offset between the minima of the features. 
We use this method to fit H-$\alpha$ ($\lambda$ 6561), O I ($\lambda$ 7774), FeII ($\lambda$ 5169), and Ca II ($\lambda\lambda\lambda$ 8498, 8542, 8662). 

We use both methods to fit Na I ($\lambda$ 5898) and H-$\beta$ ($\lambda$4861). 
For Na I, we find the spline fit does not characterize the minimum of the profile well and instead use the Gaussian minimum. 
The H-$\beta$ profile is contaminated by another feature (possibly Ti II) at later times, offsetting the minimum found using the spline fit. 
For this reason we prefer the Gaussian fit for H-$\beta$ as well. 
A comparison of the results of both fits for H-$\beta$ at early times finds them in good agreement, giving us confidence in the consistency of the two methods.

Figure \ref{fig:velocity} shows the velocities for all good fits to the data. 
The distribution of velocities is typical of other IIP/IIL supernovae representing the distribution of elements throughout the ejecta. 
A comparison of the velocities derived in this paper to the average values of 122 IIP/IIL SNe described in \citet{2017gutierrez} is shown in Figure \ref{fig:VelocityCompare}. 
The shaded regions represent the standard deviation of the sample.
We find throughout the ejecta above average velocities.
%=======================
\begin{figure}
\begin{center}
\includegraphics[width=\columnwidth]{line_velocity}
\caption{The evolution of the velocity of H-$\alpha$, H-$\beta$, Na I, O I, Ca II, Sc II, and FeII over time. 
The spread of velocities represents the different locations of the ions in the ejecta.}
\label{fig:velocity}
\end{center}
\end{figure}
%-----------------------------------------
%=======================
\begin{figure*}
\begin{center}
\includegraphics[width=\textwidth]{velocity_compare_guitierrez}
\caption{A comparison of the velocity of ASASSN-15oz to the mean velocity of 122 IIP/IIL supernovae \citep{2017gutierrez} for H-$\alpha$ (left), H-$\beta$ (middle), and Fe II (right).
For all lines the velocity is about 1$\sigma$ above average. 
Each of these line originates in a different part of the ejecta indicating that this is a global trend and the explosion energy is above average. 
Following \citep{2012silverman} we select velocity errors of 2 $\AA$, these are contained within the symbols and so are not plotted.}
\label{fig:VelocityCompare}
\end{center}
\end{figure*}
%-----------------------------------------
\subsection{Infrared Evolution}
Line identification was performed by comparing the NIR SOFI and SpeX spectra to those in \citet{2015valenti} and \citet{2018tomasella}.
The first spectrum taken 8 days post explosion (September 05, 2018) follows the optical showing strong Paschen series lines.
The later spectra, taken $\sim$40 days post explosion also mirror the optical evolution, developing metal lines as the ejecta slows and cools.
%=======================
\begin{figure*}
\begin{center}
\includegraphics[width=\textwidth]{ir_spec_montage_log}
\caption{The spectral evolution of the infrared data during the photospheric phase. 
The spectral evolution mirrors in the IR mirrors the optical evolution.
The first spectrum (dark blue) taken 8 days post-explosion shows broad hydrogen features.
The later spectra at day 38 (cyan) and and 43 (green) reveal the development metal lines.
Regions of severe sky contamination are marked with grey boxes.}
\label{fig:IRmontage}
\end{center}
\end{figure*}
%-----------------------------------------
%%%%%%%%%%%%%%%%% EVIDENCE FOR INTERACTION%%%%%%%%%%%%%%%%%%
\section{Evidence for interaction} \label{sec:Interaction}
Although it is well known that massive stars undergo significant mass-loss during their lives, the details of this mass-loss (timing, quantity) is not well understood.
The location and density of circumstellar material around a star record the amount of mass lost at different times during a star's evolution. 
As the supernova shock passes through this material, it provides a unique probe of this material and thus the mass-loss of the progenitor.
The mass loss of stars with dense CSM (e.g. type IIn SNe) that show strong evidence of interaction such as narrow emission lines in photospheric spectra or irregular photometric brightening is often studied,, the mass-loss of SNe with lower density CSM is often ignored.
In the few cases that interaction is seen in type IIP/IIL SNe, incomplete data results in a single indication of interaction (e.g. optical light curve shape or radio emission).
The linear shape of the light curve of type IIL SNe is best explained by a thin hydrogen envelope on the progenitor, implying that IIL Sne should undergo more mass-loss than IIP SNe.
ASASSN-15oz, a type IIL SNe with photometric and spectroscopic observations from the X-ray through the radio is the ideal SNe on which to study the importance of mass-loss for IIP and IIL SNe and if different observational signatures of interaction are consistent with each other. \Stefano{is there a better way to say this?}
In this section we analyze the evidence for interaction at different wavelengths.
%%%%%%%%%%%%%%%%
\subsection{UV and X-Ray}
There is no detection in any of the \textit{Swift} X-ray observation. 
To determine an upper limit, we sum all observations and find a limiting count rate of  7.234x10$^{-4}$ counts s$^{-1}$.
Assuming Galactic absorption of 6.18x$10^{20}$ cm$^{-2}$ \citep{2005kalberla}, this corresponds to an unabsorbed flux of 2.208x10$^{-14}$ erg cm$^{-2}$ s$^{-1}$ (0.4-5 keV) and a luminosity of approximately 3.72x10$^{39}$erg s$^{-1}$ at 30 Mpc.

We compare this luminosity to the model of SN 1999em in \citet{2004chugai} which has similar progenitor and explosion parameters. 
They model the interaction as an infinitely thin double-shock structure \citep{1982chevalier, 1985nadyozhin} composed of a forward shock propagating into the surrounding gas and reverse shock propagating backwards into the SN ejecta, towards its core.
\citet{2004chugai} find that the reverse shock dominates the X-ray observations for a mass loss rate of $10^{-6}$\msun/yr and a wind velocity of $10$km/s producing luminosities between $10^{38}$ and $10^{39}$ ergs/s. 
The forward shock produces luminosities comparable to the reverse shock in the first 10 days and decreasing to below $10^{34}$ ergs/s by day 100.  
Our upper limit above both the forward and reverse shock suggests that either these cannonical wind parameters must be adjusted or that a new model is needed. \Stefano{is there a better conclusion to draw?}

The \textit{Swift} grism observations show no sign of the narrow emission lines typically associated with strong interaction.
This is supported by a lack of UV continuum excess at any epoch. 
%=======================
\begin{figure*}
\begin{center}
\includegraphics[width=\textwidth]{swift_spectra} %created by make_swift_color_fig.ipynb
\caption{The Swift UV spectrum from 2015-09-05 (day 9). 
The top panel shows the spectrum taken with a roll angle of PA=248, the middle panel shows the spectrum taken with a roll angle of PA=260, and the bottom panel shows the weighted average of the flux at each roll angle.
These spectra are not template subtracted, therefore the spectra show residuals from zeroth order light. These features can be identified as being in the spectrum of one roll angle but not the other.
The lack of narrow line emission indicates a lack of strong interaction.}
\label{fig:SwiftSpectrum}
\end{center}
\end{figure*}
%-----------------------------------------
Figure \ref{fig:UVColor} shows a comparison of the UV color of ASASSN-15oz to that of all other type IIP/IIL supernovae with \textit{Swift} observations.
Each point is colored by the $S_{50V}$ slope, with shallower slopes in blue and steeper slopes in red. 
%=======================
\begin{figure}
\begin{center}
\includegraphics[width=\columnwidth]{uv_slope_comp} %created by make_swift_color_fig.ipynb
\caption{The color of all type IIP/IIL supernovae observed with \textit{Swift}. Within this sample, ASASSN-15oz shows no UV excess.}
\label{fig:UVColor}
\end{center}
\end{figure}
%-----------------------------------------
%%%%%%%%%%%%%%%%
\subsection{Radio}
The origin of radio emission from SN can be explained by interaction of the SN ejecta with the CSM. 
This interaction leads to a shockwave traveling via the CSM that in turn accelerates electrons and enhances magnetic fields and thus synchrotron emission ensues \citep{1982chevalier,1998chevalier, 2002weiler, 2006chevalier}. 
We use the \citet{1998chevalier} formalism to model the radio measurements of ASASSN 15oz. 

The \citet{1998chevalier} model assumes synchrotron self absorption (SSA). 
In such a model the radio emission peaks at a frequency below which the emission is \cancel{optically} strongly absorbed by SSA. 
Above that frequency the emission becomes optically thin. 
As the shockwave progress outwards to lower CSM densities, the SSA optical depth drops and the peak of the radio {\color{magenta}emission moves to} lower and lower frequencies. 
We use Equation 1 from \citet{1998chevalier} to model the radio dataset and derive the following properties: the shockwave velocity (assuming constant velocity) is $\approx 1.4$\,km\,s$^{-1}$. 
The CSM surrounding the SN is assumed to be created by a stellar wind from the progenitor prior to explosion with a mass-loss rate of $\dot{M}$ $\approx 0.9\times 10^{-6}$\,M$_{\odot}$\,yr$^{-1}$ for a wind velocity of $100$\,km\,s$^{-1}$. 
Note that the constraint is on the radiation between the wind mass-loss rate and the wind velocity and this the mass-loss rate should be scaled according to the chosen model wind velocity. 
%%%%%%%%%%%%%%%%
\subsection{Cachito Analysis} \label{sec:cachito}
\Stefano{Mention Pooly 2002 here?}
\citet{2007chugai} proposed a high velocity Hydrogen and Helium line produced by interaction the supernova ejecta with a stellar wind. 
According to their model, a notch on the blue side of the H-$\alpha$ profile should appear 40-80 days post explosion.
Although they could contrive a model with the introduction of a cold dense shell in which a similar notch was visible in H-$\beta$ their simplest model did not showed H-$\beta$ due to its low opacity in the wind.
They do however, predict absorption in HeI $\lambda10830$ 20-60 days post explosion, a line that is only excited when wind is present. 

A feature blueward of H-$\alpha$ is often present in type II SNe, however, it is unclear if this feature is due to high velocity hydrogen or to Si II ($\lambda$=6355)
and has thus been named the cachito feature to avoid association with a particular species.
\citet{2017gutierrez} searched for this feature in 122 type II SNe. 
They find the cachito feature in 70 SNe in their sample and divide the detections in early phase detections (phase < 40 days) and late phase detections (phase > 40 days).
The velocities of the cachito feature observed in the early sample are most often well matched to the FeII velocity if interpreted as SiII. 
For this reason, they interpret the cachito feature in the early sample as SiII and at late times as high velocity Hydrogen.
Figure \ref{fig:CachitoEvolve} shows the time evolution of the region surrounding H-$\alpha$ (left panel), H-$\beta$ (middle panel), and HeI (right panel).
H-$\alpha$ and H-$\beta$ are marked with a brick dotted line in the left and middle panels, respectively.
The dashed line in the left panel marks the location of the Cachito feature and in the middle panel, the expected location of the cachito feature in H-$\beta$.
The time of expected CSM interaction is marked in grey.
ASASSN-15oz shows a strong cachito feature in the first spectrum, 8 days post explosion. 
This feature strengthens until about day 15 and then fades until it is no longer visible at day 60. 
We find no evidence for a high velocity feature in H-$\beta$ nor HeI.
The HeI feature is fully blended with CI ( $\lambda 10691$ ) and Paschen-$\gamma$ and if it is present, it is impossible to identify or deblend.
%=======================
\begin{figure}
\begin{center}
\includegraphics[width=\columnwidth]{cachito_evolution}
\caption{The evolution of the cachito feature (left panel; dashed line) over time. 
The feature is visible bluewards of H-$\alpha$ (left panel; H-$\alpha$ marked with dotted line)  in the first spectrum , 8 days post explosion.
It further increase in strength for the next 10 days, then decreases until it is barely visible at 60 days post explosion.
This evolution is counter to the evolution described in \citet{2007chugai} in which the high velocity hydrogen feature becomes visible around day 40, making it unlikely that it is high velocity hydrogen.
There is no evidence of a high velocity hydrogen feature in H-$\beta$ (center panel; H-$\beta$ marked with dotted line; cachito velocity marked with dashed line) although this is not surprising given the low opacities predicted by \citet{2007chugai}.
The high velocity HeI ($\lambda=10830$) feature also predicted by the models of \citet{2007chugai} is too heavily blended with the Paschen-$\gamma$ and C I lines to be identifiable.}
\label{fig:CachitoEvolve}
\end{center}
\end{figure}
%-----------------------------------------
A comparison of the cachito velocity, when interpreted as SiII compared to the FeII velocity is plotted in Figure \ref{fig:SiVelocity}.
We find excellent agreement between the two velocities strengthening our conclusion that this feature is due to SiII rather than high velocity hydrogen.
%=======================
\begin{figure}
\begin{center}
\includegraphics[width=\columnwidth]{cachito_fe_vel_comp}
\caption{A comparison of the velocity of the Cachito feature if it is Si II ($\lambda=6355$) (blue triangles) and the velocity of the metals in the ejecta as characterized by the FeII $\lambda=5169$ lines (orange circles).
The velocities are the same indicating that this line is likely due to Si II rather than high velocity hydrogen in the circumstellar medium.}
\label{fig:SiVelocity}
\end{center}
\end{figure}
%-----------------------------------------

%%%%%%%%%%%%%%%%% LIGHT CURVE MODELING%%%%%%%%%%%%%%%%%%
\section{Light Curve Modeling with SNEC}
We fit the light curve of ASASSN-15oz using the Supernova Explosion Code (SNEC) \citep{2015morozova}.
SNEC is an open source Lagrangian 1D radiation hydrodynamic code that employs flux-limited radiation diffusion and assumes local thermodynamic equilibrium (LTE).
Recently, Paxton et al. 2018 demonstrated that Type II supernovae are well characterized by the radiation diffusion approximation from shock breakout through the fall from plateau.
At the same time, they have shown that the assumption of LTE is not well satisfied at the photosphere location of the Type II supernovae models. 
For this reason, SNEC color light curves generally rise faster than the light curves obtained from more sophisticated multi-group radiation-hydrodynamics codes, like STELLA (Paxton et al. 2018). 
However, this fact makes the case for introducing a CSM surrounding the RSG before its explosion even stronger.
In this view, the total CSM mass derived in our analysis may be considered as a lower limit, while the models obtained with more advanced codes need comparable, or in some cases larger values of the CSM mass \citep{2017moriya,2018paxton}.

SNEC uses a Paczy\'nski equation of state \citep{1983paczynski} solving for the ionization fractions using the Saha equations in the non-degenerate approximation \citep{2000zaghloul}. 
Opacities are drawn from OPAL Type II opacity tables \citep{1996iglesias} at high temperatures ($T=10^{4.5}-10^{8.7}$) and tables of \citet{2005ferguson} at low temperatures ($T = 10^{2.7}-10^{4.5}K$) supplemented with an opacity floor of 0.01 cm$^2$g$^{-1}$.
Mixing of isotopes due to the explosion is modeled by boxcar smoothing the progenitor composition profile with a kernel of 0.4 \msunperiod.

For the progenitor models, we use a set of non-rotating solar metallicity Red Supergiants (RSGs) parameterized by the zero-age main-sequence (ZAMS) mass M, evolved with the {\it Kepler} code \citep{1978weaver,2007woosley,2015woosley, 2014sukhbold,2016sukhbold}.
Note that the stellar evolution calculations made with {\it Kepler} take into account the regular steady winds observed in RSGs, in the way prescribed by \citet{1990nieuwenhuijzen} and \citet{1999wellstein}.
For this reason, the final pre-explosion masses of the models may be up to several $M_{\odot}$ smaller than their initial ZAMS masses.
However, for the typical mass loss rates of $\lesssim10^{-6}\,M_{\odot}\,{\rm yr}^{-1}$,  the density in these regular steady winds is too low to have any noticeable effect on the post-explosion optical light curves, and these winds are not included in the progenitor profiles. 
In the recent literature, there is a common agreement that the enhanced mass losses from the RSGs in the last years of their evolution are necessary to explain the early optical and UV observations of Type II supernovae (see, for example, \citet{2015gezari}; \citet{2017yaron}; \citet{2018bullivant}; \citet{2018foerster}). 
We explore variations in circumstellar material due to this enhanced mass-loss by adding a steady-state wind above the RSG models with density profile:
\begin{equation}
\rho(r) = \frac{\dot{M}}{4\pi r^{2}v_{wind}} = \frac{K}{r^{2}}
\end{equation}
where $\dot{M}$ is the wind mass-loss rate, $v_{wind}$ is the wind velocity. 
This approximation allows us to parameterize the CSM with two parameters, K and $R_{ext}$, the radial extent of the CSM.

%%%%%%%%%%%%%%%%
\subsection{SNEC modeling of ASASSN-15oz}\label{sec:LCmodeling}
To account for the formation of the neutron star, we excise the inner $1.4\,M_{\odot}$ of the progenitor models prior to the explosion. 
After that, we model the explosion of ASASSN-15oz using a thermal bomb with explosion energy $E_{exp}$ lasting for 1 second in the inner 0.02 \msunperiod. 
When computing the amount of energy injected in the form of a thermal bomb, SNEC automatically takes into account the total initial (negative, mostly gravitational) energy of the models. 
Therefore, by the explosion energy $E_{exp}$ in our analysis we mean the total energy of the models after explosion, which is conserved in the code to better than 1\% accuracy and eventually mainly transformed into the kinetic energy of the expanding envelope.

SNEC does not model nuclear reaction networks, but rather takes an input a mass of ${}^{56}$Ni. 
Given the uncertainty in the mass of ${}^{56}$Ni produced by ASASSN-15oz we produced models for the maximum, minimum, and mean value of synthesized ${}^{56}$Ni (see Section \ref{SecNi}).
While SNEC allows for the ${}^{56}$Ni to be mixed out to different values of mass coordinate, \citet{2017morozova} find that the progenitor masses and explosion energies derived from fitting the Type II supernova light curves are not very sensitive to the degree of ${}^{56}$Ni mixing.
Therefore, in our study we choose to keep this parameter fixed and mix the ${}^{56}$Ni up to 5.0 M$_{\odot}$.
Following \citet{2017morozova}, we explore different mass-loss scenarios by adding a constant wind density profile to RSG density profile with density K and radial extent (from the center) $R_{ext}$.

We use SNEC to find the best progenitor parameters varying the progenitor mass, explosion energy, ${}^{56}$Ni mass, CSM density, and CSM extent. 
Table \ref{tb:param} gives the set of parameters used, resulting in over 5000 model light curves.
%=======================
\begin{table}
\centering
\caption{The grid of parameters used by SNEC. 
The values that best fit the data are bolded.}
\label{tb:param}
\begin{tabular}{c|c}
Parameter & Values \\
\hline
Progenitor ZAMS Mass (\msunperiod) & 11, 13, 14, 16, 17, {\bf 18}, 21 \\
Explosion Energy (10$^{51}$ ergs) & 0.5, 0.8, 1.1, {\bf 1.4}, 1.7, 2.0 \\
CSM Density (10$^{17}$ g/cm) & {\bf 10}, 20, 30, 35, 40, 50, 60 \\
CSM Extent (100R$_{\odot}$) & 15, 18, 21, {\bf 24}, 27, 30, 33 \\
Ni Mass (\msunperiod) & {\bf 0.08}, 0.095, 0.11 \\
\end{tabular}
\end{table}
%-----------------------------------------
For ease of comparison with observations, SNEC uses the photospheric temperature at each time step to compute a blackbody spectrum, which it combines with different filter throughputs to output a light curve in sloan filters u, g, r, and i, Bessel filters U, B, V, R, and I and PanSTARRs filter z. 
During the rise and plateau phase, a blackbody should be a good approximation to the longer wavelengths. 
However, line blanketing make cause the bluer filters to be a poor representation of the observed spectrum \citep{2009kasen,2005dessart}.
For this reason the best fit model is determined using the g, r, and i filter. 
While we do have a V band light curve, the throughput heavily overlaps with the g and r bands and its inclusion would give more weight to these wavelengths without providing new information.
The best fit model is determined by interpolating the well sampled model to the observed wavelengths and computing a chi-square minimization across all three filters.
Given the uncertainty in the explosion time, we shift the model spectrum by $\pm$4 days and treat this offset as a free parameter.
We find the light curve of ASASSN-15oz is best characterized by M = 18 \msunperiod, E = 1.4x10$^{51}$ ergs, K = 10x10$^{17}$ g/cm, R = 2400 \rsun, mass$_{Ni}$ = 0.083  \msunperiod, t$_{offset}$ = -4 days.
This model is shown with and without CSM interaction in Figure \ref{fig:snecLC}. 
%=======================
\begin{figure}
\begin{center}
\includegraphics[width=\columnwidth]{lightcurve_snec} %PLACEHOLDER
\caption{The multi-band optical light curve of ASASSN-15oz. 
The observations are shown as circles and the best fit SNEC model is shown with and without CSM (solid and dashed lines respectively).
We require CSM to fit the light curve at early times. 
We expect the blue filters to be affected by line blanketing and attribute the overestimate of the model flux in the g-band to this effect.}
\label{fig:snecLC}
\end{center}
\end{figure}
%-----------------------------------------
%%%%%%%%%%%%%%%%
\subsection{Mass-loss}
There is broad degeneracy in the CSM parameters, making this analysis a better estimate of the total mass loss rather than the mass-loss history. 
Integrating the CSM density over it radial extent, we find a total mass loss of 0.62 \msunperiod. 

\citet{2018morozova} find the best fit model with and without CSM as a two step process. 
First they modeling the second half of the light curve, characterized by the S2 slope \Azalee{citation} without CSM to determine the best progenitor mass and explosion energy. 
Then, fixing explosion energy and progenitor mass, they explore the CSM parameter space. 
This is computationally less intensive than modeling the full parameter space and allows them to explore a finer grid of parameters.
Given the complexity of the parameter space, we choose to model the best light curve with CSM, exploring the full set of parameters. 
To compare our best fit model to that without CSM, we use the best fit progenitor mass and explosion energy. 
This should be a good representation of the best fit model as we expect the light curve to be the same during the S2 slope and improved by CSM at early times during the S1 slope.
%%%%%%%%%%%%%%%%
\subsection{Comparison of Observed and Modeled Ejecta Velocity}
To check the consistency of our photospheric modeling, we compare the photospheric velocity of the best fit SNEC model to the observed photospheric velocity of ASASSN-15oz (see Figure \ref{fig:SNECVelocityCompare}.
Following \citet{2014faran}, we define the photospheric velocity as that of Fe II $\lambda$5169. 
We use the relationship between the $H-\beta$ velocity and the Fe II $\lambda$5169 velocity defined in \citet{2014faran} to infer the photospheric velocity at early times when the iron line is not yet visible.
While there is fair agreement between the model and observe velocity at early times, the model velocity is significantly lower than the observed velocity starting around day 25.
This behavior was previously noted by \citet{2018paxton} who find that the Fe II $\lambda$5169 line is originating above the photosphere where the Sobolov optical depth is a specific value (they find $\tau_{sob} \sim1$ fits the observations well). 
This effect becomes significant as the photosphere begins to recede around 20-30 days exactly when we see a deviation between our model and observed velocities. 
\Azalee{decide whether to implement this or leave it as an explanation}
%=======================
\begin{figure}
\begin{center}
\includegraphics[width=\columnwidth]{snec_velocity_comp}
\caption{The model photospheric velocity computed by SNEC (blue) compared to the observed photospheric velocity (orange) taken to be the velocity of the Fe II $\lambda$5169 feature. 
The model deviated from the observations around the time that the photosphere begins to recede into the ejecta.
This discrepancy is likely the result of the Fe II line being produced above the ejecta at these times \citep{2018paxton}.
}
\label{fig:SNECVelocityCompare}
\end{center}
\end{figure}
%-----------------------------------------
%%%%%%%%%%%%%%%%% PHYSICAL PARAMETERS %%%%%%%%%%%%%%%%%%%%%
\section{Physical Parameters of ASAS-SN 15oz}
Intro text here...
%%%%%%%%%%%%%%%%
\subsection{Determining the Nickel Mass}\label{SecNi}
Following \citet{2016valenti} we find the Nickel mass synthesized in a supernova explosion by scaling the bolometric luminosity of ASASSN-15oz when it is powered by ${}^{56}Co$ to that of supernova 1987A using the following relation:
\begin{equation} \label{eqn:Ni}
M({}^{56}Ni) = 0.075M_{\odot} \times \frac{L_{sn}(t)}{L_{87A}(t)}
\end{equation}
\Stefano{does this need a reference?}
This method assumes that there is complete gamma ray trapping for at least some of the radioactive decay tail (i.e. that the slope is consistent with that expected from ${}^{56}$Co decay).
Since there is not complete trapping for any part of the observed radioactive decay tail of ASASSN-15oz, we consider the maximum and minimum Nickel mass allowed by the observations. 
For both cases we assume that the fall from plateau takes 20 days based on light curves fits presented in \citet{2016valenti}. 
An upper limit on the Nickel mass can be obtains if ASASSN-15oz fell from plateau immediately after the last photospheric point.
For this case, we construct an artificial fall from plateau, lasting 20 days, and terminating at a fit to the radioactive decay tail, extrapolated to earlier times.
Using this light curve (see top panel of Figure \ref{fig:Ni}) and assuming complete trapping at the end of the fall from plateau, we calculate a Nickel mass of 0.1 \msun by scaling the luminosity at 102 days.
To find the lower limit, we estimate a conservative value of the time to the middle of the fall from plateau ($t_{pt}$) to be 125 days from Figure 5a in \citet{2016valenti}.
Using this $t_{pt}$ and a 20 day fall from plateau, we calculate the fall from plateau to end 135 days after explosion and extrapolate the tail fit to this phase (see lower panel of Figure \ref{fig:Ni}). \Stefano{should we mention the time frame over which we fit the slope?} 
Again, assuming complete trapping at the beginning of the radioactive decay tail, we compute a Nickel mass of 0.08 \msun by scaling the luminosity at 135 days.
Finally, we consider an intermediate Nickel mass, the mean of the upper and lower limits, 0.95 \msunperiod.
%=======================
\begin{figure}
\begin{center}
\includegraphics[width=\columnwidth]{ni_mass_lc} %created by make_ni_figure.ipynb
\caption{The artificial light curve of ASASSN-15oz for the shortest (top) and longest (bottom) plateau lengths. 
The grey points represent the bolometric luminosity while the solid blue line represents a fit to the s2 data, the yellow line the artificial fall from plateau, and the red line is a fit to the radioactive decay tail.
The green star marks the point to which the light curve of SN 1987A is scaled to derive a nickel mass. 
Plotted as a cyan dashed line is the slope of the radioactive decay tail if there was complete trapping.
An upper limit on the Nickel mass of 0.1 \msun is derived from the light curve in the top plot while a lower limit of 0.08 \msun is derived from the lower plot. }
\label{fig:Ni}
\end{center}
\end{figure}
%-----------------------------------------
%%%%%%%%%%%%%%%%
\subsection{Progenitor Mass from Nebular Spectra} \label{sec:nebular}
Around 200 days the supernova ejecta becomes optically transparent, revealing the inner core. 
At this time the spectra are nebular, showing strong emission lines and no continuum (see Figure \ref{fig:neb}).
Analysis of the spectra at this time can constrain the geometry of the explosion as well as the abundance of different elements.
In particular, spectral modeling of this stage is a powerful tool to constrain the nature of the progenitor.
\citet{2012jerkstrand}, \citet{2013dessart}, and \citet{2014jerkstrand} have shown that the intensity of a few lines are sensitive to the mass of the progenitor.
Specifically, there is a tight monotonic correlation between the strength of the [OI] ($\lambda\lambda6300, 6334$) line and the mass of the progenitor.
This is because [OI] is more isolated and less sensitive to the explosive nucleosynthesis of the supernova than the other lines and thus reflects the oxygen abundance of the progenitor which has been shown to correlate well with progenitor mass. \Stefano{Which Wheeler paper did you want to cite here?}

\citet{2014jerkstrand} modeled nebular spectra for 12, 15, 19, and 25 M$_{\odot}$ progenitors (MZAMS).
Starting with the explosion models from \citet{2007woosley}, the ejecta is divided into zones based on chemical composition and the evolution of the spectrum is found by modeling the gamma ray transport and deposition, non-thermal electron degradation, NLTE ionization and excitation, and Monte Carlo radiative transfer.

Nebular spectra of ASASSN-15oz, were taken on April 11, 2016 (day 228), August 03, 2016 (day 340), and September 19, 2016 (day 388) of which the first two were high enough S/N to model.
In Figure \ref{fig:neb} we compare the first two observations to the different mass models, scaling the models to the observations over the full wavelength range. 
A zoom in on the [OI] lines are displayed in the inset figure. We find the [OI] strength falls between the 15 and 19 M$_\odot$ models consistent with the 18 M$_{\odot}$ found by modeling the light curve.

Following \citet{2018jerkstrand}, the scaling combined with the equation for the luminosity of the Cobalt decay (equation 6 of \citet{2012jerkstrand} yields the Nickel mass (assuming complete trapping): \begin{equation}
\frac{F_{obs}}{F_{mod}} = \frac{d_{mod}^{2}}{d_{obs}^{2}}\frac{(M_{56,Ni}){}_{obs}}{(M_{56, Ni}){}_{mod}}exp\left(\frac{t_{mod} - t_{obs}}{111.4}\right)
\end{equation}
Where $F_{obs}$ and $F_{mod}$ are the observed and model flux, $d_{obs}$ and $d_{mod}$ the observed and model distance, $M_{56, Ni}{}_{obs}$ and $M_{56, Ni}{}_{mod}$ are the observed and model Nickel mass, and $t_{obs}$ and $t_{mod}$ are epoch of the observation and the model.
Using this relation we find a Nickel mass of 0.02 \msunperiod.
Given the constraints placed on the Nickel mass in Section \ref{SecNi} and the fact the ASASSN-15oz has incomplete gamma ray trapping, we find this scaling reasonable but cannot use it to set further limits on the Nickel mass.  
%=======================
\begin{figure}
\begin{center}
\includegraphics[width=\columnwidth]{nebular_spectra_OI} %created by make_nebular_figure.ipynb
\caption{A nebular spectrum of ASASSN-15oz from April 11, 2015 (day 228; top) and August 03, 2016 (day 340; bottom) plotted with the scaled models of \citet{2014jerkstrand} for MZAMS = 12, 15, 19, and 25 M$_{\odot}$. 
Most features in the observed spectrum well matched by the models. 
The inset shows the [O I] ($\lambda\lambda6300, 6334$) doublet, the strength of which increases monotonically with progenitor mass. 
The [O I] flux falls between the 15 and 19 M$_{\odot}$ models, consistent the 18 M$_{\odot}$ progenitor mass found using light curve modeling.}
\label{fig:neb}
\end{center}
\end{figure}
%-----------------------------------------
%%%%%%%%%%%%%%%%% DISCUSSION %%%%%%%%%%%%%%%%%%%%%
\section{Discussion and Summary} \label{SecComp}
To find objects with a similar light curve to ASASSN-15oz we compare the $s_{50V}$ slope to that of other supernovae \Stefano{how should we describe the database?}.
SN 2016zb and SN 2015W have the best matched slopes. We compare the multi-band light curves and spectra of these objects and find them to be very similar in all respects.

\Azalee{Revisit this figure. Is V-band the right one to measure both of these? how complete is the database? Consider cutting those with too large slopes}

High velocity consistent with a larger than average explosion energy \Stefano{and mass?}. 
High velocity is consistent with the light curve modeling (see Section \ref{sec:LCmodeling}).
We also find incomplete gamma ray trapping during the nebular phase (see Section \ref{sec:LCEvolve}) which is anticipated in supernovae with high core velocity \ref{2011jerkstrand}.
%=======================
\begin{figure}
\begin{center}
\includegraphics[width=\columnwidth]{slope_compare} %created by lightcurve_characteristics.ipynb
\caption{Center: The slope of the light curve at late times, during the radioactive decay phase compared to the slope during the photospheric phase for a sample of publicly available supernova light curves.
ASASSN-15oz is plotted in red.
Top (Right): A histogram of the radioactive decay (photometric) phase slopes. All measured slopes from this phase are included in the hatched sample.
The objects for which there is both a decay phase slope and a photospheric slope are plotted in blue. The comparison of the full slope sample (hatch) to the combined slope sample (blue) gives us confidence that points we plot in the center are representative of the full sample.}
\label{fig:SlopeComp}
\end{center}
\end{figure}
%-----------------------------------------
\section*{Acknowlegments}
This work is based (in part) on observations collected at the European Organization for Astronomical Research in the Southern Hemisphere, Chile as part of PESSTO, (the Public ESO Spectroscopic Survey for Transient Objects Survey) ESO program 188.D-3003, 191.D-0935, 197.D-1075.
%\software{Astropy}

\bibliographystyle{mnras}
\bibliography{references}
%%%%%%%%%%%%%%%%%%%%%%%%%%%%%%%%%%%%%%%%%%%%%%%%%%

%%%%%%%%%%%%%%%%% APPENDICES %%%%%%%%%%%%%%%%%%%%%
\newpage
\clearpage
\appendix
\section{Line Fitting Details}\label{AppLineFit}
We eliminate large noise spikes (e.g. from cosmic rays) using a Savitzky-Golay smoothing filter \citep{1964savitzky} with a quadratic function over a binsize of 5 pixels. 
For each line, a minimum and maximum wavelength is defined as well as a slope to account for the shifting of the feature over time as the ejecta slows. 
A slope is then fit to small bins over this wavelength range, starting with a binsize of 5 pixels and increasing until exactly 3 changes in slope are found (at either edge of the line and the line center). 
If the binsize reaches 40\% of the feature size, no further attempt at fitting is made. 
The feature edges are confirmed by fitting a quadratic function to the region centered on each edge identified by the slope change with a width of 20 pixels for the FLOYDS spectra and 5 pixels for the EFOSC spectra. 
The edge is considered successfully found if the quadratic fit is concave down. 
The continuum is defined by a line though these end points.
Finally, the velocity is found by fitting a cubic spline to the continuum subtracted flux between the edges using the flux errors as weights. 
%=======================
\begin{figure*}
\begin{center}
\includegraphics[width=\textwidth]{example_fits}
\caption{An example of a fit to the multi-component Ca II NIR triplet (left) using multiple Gaussians and a single line fit (following \citet{2012silverman}, right) to the Sc II $\lambda$5662 line.
The observed spectrum is plotted in blue.
The continuum edges are marked with a red circle and the red dashed line connecting these points is used as the continuum. 
The Ca II NIR triplet fit is found by simultaneously fitting 3 Gaussians with the same standard deviation and mean offsets corresponding to the expected wavelength separation of the triplet. 
The individual Gaussians are plotting at black dotted lines and the combined fit is plotted as a solid black line. 
The minima of the individual Gaussians is used to find the velocity of each component.
The Sc II feature is fit with a cubic spliine. 
The minimum of the spline is used to find the Sc II velocity.
}
\label{fig:VelocityFit}
\end{center}
\end{figure*}
%-----------------------------------------
%Consider making this two columns
%Make an epoch column
\section{Observational Data}
%=======================
\begin{table*}
\caption{Spectroscopic Observations of ASASSN-15oz.\label{tab:SpecObs}}
\begin{tabular}{ccccc}
\hline
Date & JD & Phase (Day) & Observatory & Instrument \\
\hline
2015-09-04 & 2457270.0 & 8.0 & LCO & FLOYDS \\
2015-09-05 & 2457270.7 & 8.7 & Swift & UVOTA \\
2015-09-05 & 2457271.1 & 9.1 & Swift & UVOTA \\
2015-09-05 & 2457270.6 & 8.6 & NTT & SOFI \\
2015-09-06 & 2457272.0 & 10.0 & LCO & FLOYDS \\
2015-09-07 & 2457273.0 & 11.0 & LCO & FLOYDS \\
2015-09-10 & 2457275.7 & 13.7 & Swift & UVOTA \\
2015-09-10 & 2457275.7 & 13.7 & Swift & UVOTA \\
2015-09-11 & 2457277.0 & 15.0 & LCO & FLOYDS \\
2015-09-11 & 2457276.9 & 14.9 & Swift & UVOTA \\
2015-09-16 & 2457282.0 & 20.0 & LCO & FLOYDS \\
2015-09-20 & 2457286.1 & 24.1 & LCO & FLOYDS \\
2015-09-21 & 2457287.5 & 25.5 & VLT & X-SHOOTER \\
2015-09-24 & 2457290.0 & 28.0 & LCO & FLOYDS \\
2015-09-30 & 2457296.0 & 34.0 & LCO & FLOYDS \\
2015-10-04 & 2457299.5 & 37.5 & NTT & EFOSC \\
2015-10-05 & 2457300.5 & 38.5 & NTT & SOFI \\
2015-10-06 & 2457301.9 & 39.9 & LCO & FLOYDS \\
2015-10-10 & 2457305.7 & 43.7 & IRTF & SpeX \\
2015-10-14 & 2457310.0 & 48.0 & LCO & FLOYDS \\
2015-10-25 & 2457320.9 & 58.9 & LCO & FLOYDS \\
2015-11-07 & 2457333.9 & 71.9 & LCO & FLOYDS \\
2015-11-08 & 2457334.5 & 72.5 & NTT & EFOSC \\
2015-11-19 & 2457345.5 & 83.5 & NTT & EFOSC \\
2016-04-11 & 2457489.9 & 227.9 & NTT & EFOSC \\
2016-04-11 & 2457489.9 & 227.9 & NTT & EFOSC \\
2016-06-09 & 2457548.8 & 286.8 & Gemini-S & GMOS \\
2016-06-10 & 2457549.7 & 287.7 & Gemini-S & GMOS \\
2016-06-12 & 2457551.7 & 289.7 & Gemini-S & GMOS \\
2016-08-03 & 2457603.7 & 341.7 & NTT & EFOSC \\
2016-09-11$\rm{^{*}}$ & 2457642.6 & 380.6 & NTT & EFOSC \\
2016-09-11$\rm{^{*}}$ & 2457642.6 & 380.6 & NTT & EFOSC \\
2016-09-19 & 2457650.5 & 388.5 & NTT & EFOSC \\
2017-09-20$\rm{^{+}}$ & 2458016.9 & 754.9 & Swift & UVOTA \\
\hline
\end{tabular}
\\$\rm{^{*}}$No signal in data due to cloud cover\\ $\rm{^{+}}$ Template observation
\end{table*}
 \label{tab:SpecObs}
%-----------------------------------------
\newpage
%=======================
\Azalee{Add swift XRT observations?}
\begin{deluxetable}{cccccc} 
\tablecolumns{6} 
\tablecaption{ Imaging Observations } 
\tablehead{ 
\colhead{Date-Obs} & \colhead{JD} & \colhead{Apparent Magnitude} &  \colhead{Apparent Magnitude Error} & \colhead{Filter} & \colhead{Facility}\\
}
\startdata 
2015-09-04 & 2457269.68 & 14.64 & 0.02 & B & LSC 1m \\
2015-09-04 & 2457269.68 & 14.66 & 0.02 & B & LSC 1m \\
2015-09-04 & 2457269.69 & 14.37 & 0.02 & V & LSC 1m \\
2015-09-04 & 2457269.69 & 14.43 & 0.01 & g & LSC 1m \\
2015-09-04 & 2457269.69 & 14.42 & 0.01 & g & LSC 1m \\
2015-09-04 & 2457269.70 & 14.49 & 0.01 & r & LSC 1m \\
2015-09-04 & 2457269.70 & 14.45 & 0.01 & r & LSC 1m \\
2015-09-04 & 2457269.70 & 14.51 & 0.01 & i & LSC 1m \\
2015-09-04 & 2457269.70 & 14.47 & 0.01 & i & LSC 1m \\
2015-09-04 & 2457269.93 & 14.30 & 0.06 & uvw1 & Swift \\
2015-09-04 & 2457269.93 & 14.10 & 0.05 & u & Swift \\
2015-09-04 & 2457269.93 & 14.73 & 0.05 & b & Swift \\
2015-09-04 & 2457269.93 & 15.26 & 0.07 & uvw2 & Swift \\
2015-09-04 & 2457269.93 & 14.68 & 0.05 & v & Swift \\
2015-09-04 & 2457269.94 & 14.91 & 0.06 & uvm2 & Swift \\
2015-09-04 & 2457270.09 & 14.60 & 0.02 & B & COJ 1m \\
2015-09-04 & 2457270.09 & 14.64 & 0.02 & B & COJ 1m \\
2015-09-04 & 2457270.09 & 14.46 & 0.02 & V & COJ 1m \\
2015-09-04 & 2457270.09 & 14.46 & 0.02 & V & COJ 1m \\
2015-09-04 & 2457270.10 & 14.50 & 0.01 & g & COJ 1m \\
2015-09-04 & 2457270.10 & 14.51 & 0.01 & g & COJ 1m \\
2015-09-04 & 2457270.10 & 14.44 & 0.02 & r & COJ 1m \\
2015-09-04 & 2457270.10 & 14.46 & 0.01 & r & COJ 1m \\
2015-09-04 & 2457270.10 & 14.47 & 0.01 & i & COJ 1m \\
2015-09-04 & 2457270.11 & 14.47 & 0.01 & i & COJ 1m \\
2015-09-04 & 2457270.44 & 14.71 & 0.02 & B & CPT 1m \\
2015-09-04 & 2457270.44 & 14.71 & 0.02 & B & CPT 1m \\
2015-09-04 & 2457270.44 & 14.62 & 0.02 & V & CPT 1m \\
2015-09-04 & 2457270.45 & 14.51 & 0.01 & g & CPT 1m \\
2015-09-04 & 2457270.45 & 14.54 & 0.01 & g & CPT 1m \\
2015-09-04 & 2457270.45 & 14.50 & 0.01 & r & CPT 1m \\
2015-09-04 & 2457270.45 & 14.52 & 0.01 & r & CPT 1m \\
2015-09-04 & 2457270.46 & 14.57 & 0.02 & i & CPT 1m \\
2015-09-04 & 2457270.46 & 14.60 & 0.02 & B & CPT 1m \\
2015-09-04 & 2457270.47 & 14.59 & 0.02 & B & CPT 1m \\
2015-09-05 & 2457270.68 & 14.73 & 0.02 & B & LSC 1m \\
2015-09-05 & 2457270.68 & 14.73 & 0.02 & B & LSC 1m \\
2015-09-05 & 2457270.69 & 14.48 & 0.02 & V & LSC 1m \\
2015-09-05 & 2457270.69 & 14.52 & 0.01 & g & LSC 1m \\
2015-09-05 & 2457270.70 & 14.48 & 0.01 & r & LSC 1m \\
2015-09-05 & 2457270.70 & 14.40 & 0.02 & i & LSC 1m \\
2015-09-05 & 2457270.80 & 14.41 & 0.06 & uvw1 & Swift \\
2015-09-05 & 2457270.82 & 15.21 & 0.08 & uvw2 & Swift \\
2015-09-05 & 2457271.10 & 14.09 & 0.12 & U & COJ 1m \\
2015-09-05 & 2457271.10 & 14.15 & 0.12 & U & COJ 1m \\
2015-09-05 & 2457271.10 & 14.72 & 0.02 & B & COJ 1m \\
2015-09-05 & 2457271.11 & 14.71 & 0.02 & B & COJ 1m \\
2015-09-05 & 2457271.11 & 14.53 & 0.02 & V & COJ 1m \\
2015-09-05 & 2457271.11 & 14.53 & 0.02 & V & COJ 1m \\
2015-09-05 & 2457271.11 & 14.48 & 0.01 & g & COJ 1m \\
2015-09-05 & 2457271.11 & 14.48 & 0.01 & g & COJ 1m \\
2015-09-05 & 2457271.12 & 14.45 & 0.01 & r & COJ 1m \\
2015-09-05 & 2457271.12 & 14.46 & 0.01 & r & COJ 1m \\
2015-09-05 & 2457271.12 & 15.27 & 0.08 & uvw2 & Swift \\
2015-09-05 & 2457271.12 & 14.50 & 0.01 & i & COJ 1m \\
2015-09-05 & 2457271.12 & 14.54 & 0.01 & i & COJ 1m \\
2015-09-05 & 2457271.13 & 14.52 & 0.06 & uvw1 & Swift \\
2015-09-05 & 2457271.45 & 13.99 & 0.18 & U & CPT 1m \\
2015-09-05 & 2457271.45 & 14.15 & 0.09 & U & CPT 1m \\
2015-09-05 & 2457271.46 & 14.69 & 0.02 & B & CPT 1m \\
2015-09-05 & 2457271.46 & 14.71 & 0.02 & B & CPT 1m \\
2015-09-05 & 2457271.46 & 14.51 & 0.02 & V & CPT 1m \\
2015-09-05 & 2457271.46 & 14.52 & 0.02 & V & CPT 1m \\
2015-09-05 & 2457271.46 & 14.50 & 0.01 & g & CPT 1m \\
2015-09-05 & 2457271.47 & 14.49 & 0.01 & g & CPT 1m \\
2015-09-05 & 2457271.47 & 14.42 & 0.01 & r & CPT 1m \\
2015-09-05 & 2457271.47 & 14.43 & 0.01 & r & CPT 1m \\
2015-09-05 & 2457271.47 & 14.45 & 0.01 & i & CPT 1m \\
2015-09-05 & 2457271.47 & 14.47 & 0.01 & i & CPT 1m \\
2015-09-06 & 2457272.45 & 14.15 & 0.15 & U & CPT 1m \\
2015-09-06 & 2457272.46 & 14.72 & 0.02 & B & CPT 1m \\
2015-09-06 & 2457272.46 & 14.50 & 0.02 & V & CPT 1m \\
2015-09-06 & 2457272.46 & 14.51 & 0.02 & V & CPT 1m \\
2015-09-06 & 2457272.46 & 14.51 & 0.01 & g & CPT 1m \\
2015-09-06 & 2457272.46 & 14.51 & 0.01 & g & CPT 1m \\
2015-09-06 & 2457272.47 & 14.43 & 0.01 & r & CPT 1m \\
2015-09-06 & 2457272.47 & 14.44 & 0.01 & r & CPT 1m \\
2015-09-06 & 2457272.47 & 14.51 & 0.01 & i & CPT 1m \\
2015-09-07 & 2457273.09 & 14.74 & 0.01 & B & COJ 1m \\
2015-09-07 & 2457273.09 & 14.73 & 0.01 & B & COJ 1m \\
2015-09-07 & 2457273.09 & 14.52 & 0.01 & V & COJ 1m \\
2015-09-07 & 2457273.10 & 14.49 & 0.01 & V & COJ 1m \\
2015-09-07 & 2457273.10 & 14.50 & 0.01 & g & COJ 1m \\
2015-09-07 & 2457273.10 & 14.48 & 0.01 & g & COJ 1m \\
2015-09-07 & 2457273.10 & 14.45 & 0.01 & r & COJ 1m \\
2015-09-07 & 2457273.11 & 14.44 & 0.01 & i & COJ 1m \\
2015-09-07 & 2457273.11 & 14.46 & 0.01 & i & COJ 1m \\
2015-09-08 & 2457274.43 & 14.51 & 0.02 & V & CPT 1m \\
2015-09-08 & 2457274.43 & 14.21 & 0.02 & R & CPT 1m \\
2015-09-08 & 2457274.45 & 14.67 & 0.02 & B & CPT 1m \\
2015-09-08 & 2457274.46 & 14.82 & 0.05 & B & CPT 1m \\
2015-09-08 & 2457274.46 & 14.45 & 0.02 & V & CPT 1m \\
2015-09-08 & 2457274.46 & 14.53 & 0.02 & V & CPT 1m \\
2015-09-08 & 2457274.46 & 14.57 & 0.01 & g & CPT 1m \\
2015-09-08 & 2457274.46 & 14.59 & 0.01 & g & CPT 1m \\
2015-09-08 & 2457274.47 & 14.42 & 0.01 & r & CPT 1m \\
2015-09-08 & 2457274.47 & 14.43 & 0.01 & r & CPT 1m \\
2015-09-08 & 2457274.47 & 14.48 & 0.01 & i & CPT 1m \\
2015-09-08 & 2457274.47 & 14.50 & 0.01 & i & CPT 1m \\
2015-09-09 & 2457275.33 & 15.22 & 0.08 & uvw1 & Swift \\
2015-09-09 & 2457275.34 & 14.38 & 0.06 & u & Swift \\
2015-09-09 & 2457275.34 & 14.93 & 0.06 & b & Swift \\
2015-09-09 & 2457275.34 & 16.13 & 0.09 & uvw2 & Swift \\
2015-09-09 & 2457275.34 & 14.62 & 0.07 & v & Swift \\
2015-09-09 & 2457275.34 & 16.05 & 0.08 & uvm2 & Swift \\
2015-09-10 & 2457275.74 & 16.50 & 0.12 & uvw2 & Swift \\
2015-09-10 & 2457275.89 & 15.44 & 0.06 & uvw1 & Swift \\
2015-09-10 & 2457275.96 & 15.42 & 0.06 & uvw1 & Swift \\
2015-09-10 & 2457275.97 & 16.30 & 0.09 & uvw2 & Swift \\
2015-09-10 & 2457276.06 & 14.52 & 0.27 & U & COJ 1m \\
2015-09-10 & 2457276.07 & 14.83 & 0.02 & B & COJ 1m \\
2015-09-10 & 2457276.08 & 14.52 & 0.02 & V & COJ 1m \\
2015-09-10 & 2457276.08 & 14.51 & 0.02 & V & COJ 1m \\
2015-09-10 & 2457276.08 & 14.55 & 0.01 & g & COJ 1m \\
2015-09-10 & 2457276.08 & 14.56 & 0.01 & g & COJ 1m \\
2015-09-10 & 2457276.08 & 14.40 & 0.01 & r & COJ 1m \\
2015-09-10 & 2457276.09 & 14.35 & 0.01 & r & COJ 1m \\
2015-09-10 & 2457276.09 & 14.38 & 0.01 & i & COJ 1m \\
2015-09-10 & 2457276.09 & 14.40 & 0.01 & i & COJ 1m \\
2015-09-11 & 2457276.88 & 14.75 & 0.05 & u & Swift \\
2015-09-11 & 2457276.90 & 16.50 & 0.10 & uvw2 & Swift \\
2015-09-11 & 2457277.06 & 14.79 & 0.05 & u & Swift \\
2015-09-11 & 2457277.07 & 16.75 & 0.09 & uvw2 & Swift \\
2015-09-12 & 2457277.70 & 15.03 & 0.02 & B & LSC 1m \\
2015-09-12 & 2457277.70 & 14.57 & 0.01 & g & LSC 1m \\
2015-09-12 & 2457277.71 & 14.57 & 0.01 & g & LSC 1m \\
2015-09-12 & 2457277.71 & 14.32 & 0.01 & r & LSC 1m \\
2015-09-12 & 2457277.71 & 14.36 & 0.01 & i & LSC 1m \\
2015-09-14 & 2457280.41 & 15.11 & 0.02 & B & CPT 1m \\
2015-09-14 & 2457280.41 & 15.11 & 0.02 & B & CPT 1m \\
2015-09-14 & 2457280.41 & 14.54 & 0.02 & V & CPT 1m \\
2015-09-14 & 2457280.41 & 14.51 & 0.07 & V & CPT 1m \\
2015-09-14 & 2457280.41 & 14.87 & 0.07 & g & CPT 1m \\
2015-09-14 & 2457280.42 & 14.75 & 0.02 & g & CPT 1m \\
2015-09-14 & 2457280.45 & 15.07 & 0.04 & B & CPT 1m \\
2015-09-15 & 2457281.27 & 15.11 & 0.02 & B & CPT 1m \\
2015-09-15 & 2457281.27 & 15.11 & 0.02 & B & CPT 1m \\
2015-09-15 & 2457281.27 & 14.55 & 0.02 & V & CPT 1m \\
2015-09-15 & 2457281.27 & 14.60 & 0.02 & V & CPT 1m \\
2015-09-15 & 2457281.27 & 14.73 & 0.02 & g & CPT 1m \\
2015-09-15 & 2457281.28 & 14.75 & 0.02 & g & CPT 1m \\
2015-09-15 & 2457281.28 & 14.36 & 0.02 & r & CPT 1m \\
2015-09-15 & 2457281.28 & 14.32 & 0.02 & r & CPT 1m \\
2015-09-15 & 2457281.28 & 14.33 & 0.01 & i & CPT 1m \\
2015-09-15 & 2457281.29 & 14.82 & 0.23 & U & CPT 1m \\
2015-09-15 & 2457281.30 & 14.94 & 0.17 & U & CPT 1m \\
2015-09-15 & 2457281.30 & 15.16 & 0.02 & B & CPT 1m \\
2015-09-15 & 2457281.30 & 15.04 & 0.03 & B & CPT 1m \\
2015-09-15 & 2457281.30 & 14.56 & 0.02 & V & CPT 1m \\
2015-09-15 & 2457281.31 & 14.51 & 0.03 & V & CPT 1m \\
2015-09-15 & 2457281.31 & 14.82 & 0.02 & g & CPT 1m \\
2015-09-15 & 2457281.31 & 14.78 & 0.02 & g & CPT 1m \\
2015-09-15 & 2457281.31 & 14.37 & 0.02 & r & CPT 1m \\
2015-09-15 & 2457281.32 & 14.39 & 0.02 & r & CPT 1m \\
2015-09-15 & 2457281.32 & 14.31 & 0.02 & i & CPT 1m \\
2015-09-15 & 2457281.32 & 14.32 & 0.01 & i & CPT 1m \\
2015-09-15 & 2457281.38 & 16.57 & 0.10 & uvw1 & Swift \\
2015-09-15 & 2457281.38 & 15.35 & 0.07 & u & Swift \\
2015-09-15 & 2457281.38 & 15.31 & 0.06 & b & Swift \\
2015-09-15 & 2457281.39 & 17.88 & 0.14 & uvw2 & Swift \\
2015-09-15 & 2457281.39 & 14.63 & 0.06 & v & Swift \\
2015-09-15 & 2457281.39 & 17.70 & 0.17 & uvm2 & Swift \\
2015-09-15 & 2457281.40 & 14.56 & 0.02 & V & CPT 1m \\
2015-09-15 & 2457281.40 & 14.15 & 0.02 & R & CPT 1m \\
2015-09-16 & 2457281.58 & 16.41 & 0.08 & uvw1 & Swift \\
2015-09-16 & 2457281.58 & 15.26 & 0.06 & u & Swift \\
2015-09-16 & 2457281.58 & 15.12 & 0.05 & b & Swift \\
2015-09-16 & 2457281.58 & 17.47 & 0.11 & uvw2 & Swift \\
2015-09-16 & 2457281.58 & 14.62 & 0.05 & v & Swift \\
2015-09-16 & 2457281.59 & 17.66 & 0.11 & uvm2 & Swift \\
2015-09-16 & 2457282.24 & 14.55 & 0.02 & V & CPT 1m \\
2015-09-16 & 2457282.25 & 14.53 & 0.02 & V & CPT 1m \\
2015-09-16 & 2457282.25 & 14.21 & 0.02 & R & CPT 1m \\
2015-09-16 & 2457282.25 & 15.19 & 0.02 & B & CPT 1m \\
2015-09-17 & 2457283.04 & 14.52 & 0.02 & V & COJ 1m \\
2015-09-17 & 2457283.04 & 14.16 & 0.02 & R & COJ 1m \\
2015-09-17 & 2457283.05 & 15.17 & 0.02 & B & COJ 1m \\
2015-09-18 & 2457283.95 & 14.07 & 0.03 & R & COJ 2m \\
2015-09-18 & 2457283.95 & 13.83 & 0.06 & I & COJ 2m \\
2015-09-18 & 2457284.00 & 14.17 & 0.03 & R & COJ 1m \\
2015-09-18 & 2457284.00 & 15.18 & 0.05 & B & COJ 1m \\
2015-09-19 & 2457284.90 & 16.99 & 0.08 & uvw1 & Swift \\
2015-09-19 & 2457284.90 & 15.69 & 0.06 & u & Swift \\
2015-09-19 & 2457284.90 & 15.38 & 0.05 & b & Swift \\
2015-09-19 & 2457284.90 & 18.09 & 0.10 & uvw2 & Swift \\
2015-09-19 & 2457284.90 & 14.67 & 0.04 & v & Swift \\
2015-09-19 & 2457284.90 & 18.39 & 0.10 & uvm2 & Swift \\
2015-09-19 & 2457285.03 & 14.61 & 0.02 & V & COJ 1m \\
2015-09-19 & 2457285.03 & 14.16 & 0.02 & R & COJ 1m \\
2015-09-19 & 2457285.03 & 15.26 & 0.02 & B & COJ 1m \\
2015-09-19 & 2457285.40 & 14.94 & 0.30 & U & CPT 1m \\
2015-09-19 & 2457285.40 & 15.36 & 0.02 & B & CPT 1m \\
2015-09-19 & 2457285.41 & 15.36 & 0.02 & B & CPT 1m \\
2015-09-19 & 2457285.41 & 14.60 & 0.02 & V & CPT 1m \\
2015-09-19 & 2457285.41 & 14.62 & 0.02 & V & CPT 1m \\
2015-09-19 & 2457285.41 & 14.90 & 0.01 & g & CPT 1m \\
2015-09-19 & 2457285.42 & 14.93 & 0.01 & g & CPT 1m \\
2015-09-19 & 2457285.42 & 14.35 & 0.01 & r & CPT 1m \\
2015-09-19 & 2457285.42 & 14.35 & 0.01 & r & CPT 1m \\
2015-09-19 & 2457285.42 & 14.33 & 0.01 & i & CPT 1m \\
2015-09-20 & 2457285.90 & 14.51 & 0.06 & V & COJ 2m \\
2015-09-20 & 2457285.90 & 14.10 & 0.03 & R & COJ 2m \\
2015-09-20 & 2457285.90 & 13.84 & 0.06 & I & COJ 2m \\
2015-09-20 & 2457286.38 & 15.41 & 0.02 & B & CPT 1m \\
2015-09-20 & 2457286.38 & 15.42 & 0.02 & B & CPT 1m \\
2015-09-20 & 2457286.38 & 14.65 & 0.02 & V & CPT 1m \\
2015-09-20 & 2457286.38 & 14.62 & 0.02 & V & CPT 1m \\
2015-09-20 & 2457286.39 & 14.95 & 0.01 & g & CPT 1m \\
2015-09-20 & 2457286.39 & 14.95 & 0.01 & g & CPT 1m \\
2015-09-20 & 2457286.39 & 14.35 & 0.01 & r & CPT 1m \\
2015-09-20 & 2457286.39 & 14.35 & 0.01 & r & CPT 1m \\
2015-09-20 & 2457286.39 & 14.32 & 0.01 & i & CPT 1m \\
2015-09-20 & 2457286.40 & 14.33 & 0.01 & i & CPT 1m \\
2015-09-21 & 2457286.92 & 14.51 & 0.07 & V & COJ 2m \\
2015-09-21 & 2457286.93 & 14.10 & 0.04 & R & COJ 2m \\
2015-09-21 & 2457286.93 & 13.86 & 0.07 & I & COJ 2m \\
2015-09-21 & 2457287.03 & 14.62 & 0.02 & V & COJ 1m \\
2015-09-21 & 2457287.03 & 14.20 & 0.02 & R & COJ 1m \\
2015-09-21 & 2457287.03 & 15.45 & 0.02 & B & COJ 1m \\
2015-09-22 & 2457288.00 & 14.61 & 0.02 & V & COJ 1m \\
2015-09-22 & 2457288.00 & 14.22 & 0.02 & R & COJ 1m \\
2015-09-22 & 2457288.00 & 15.48 & 0.02 & B & COJ 1m \\
2015-09-23 & 2457288.74 & 14.62 & 0.06 & V & OGG 2m \\
2015-09-23 & 2457288.74 & 14.14 & 0.03 & R & OGG 2m \\
2015-09-23 & 2457289.00 & 14.68 & 0.02 & V & COJ 1m \\
2015-09-23 & 2457289.00 & 14.24 & 0.02 & R & COJ 1m \\
2015-09-23 & 2457289.00 & 15.51 & 0.02 & B & COJ 1m \\
2015-09-23 & 2457289.02 & 14.58 & 0.07 & V & COJ 2m \\
2015-09-23 & 2457289.02 & 14.15 & 0.03 & R & COJ 2m \\
2015-09-23 & 2457289.02 & 13.86 & 0.06 & I & COJ 2m \\
2015-09-23 & 2457289.27 & 15.54 & 0.02 & B & CPT 1m \\
2015-09-23 & 2457289.28 & 15.54 & 0.02 & B & CPT 1m \\
2015-09-23 & 2457289.28 & 14.63 & 0.02 & V & CPT 1m \\
2015-09-23 & 2457289.28 & 14.63 & 0.02 & V & CPT 1m \\
2015-09-23 & 2457289.28 & 15.12 & 0.01 & g & CPT 1m \\
2015-09-23 & 2457289.28 & 15.06 & 0.01 & g & CPT 1m \\
2015-09-23 & 2457289.29 & 14.38 & 0.01 & r & CPT 1m \\
2015-09-23 & 2457289.29 & 14.38 & 0.01 & r & CPT 1m \\
2015-09-23 & 2457289.29 & 14.35 & 0.01 & i & CPT 1m \\
2015-09-23 & 2457289.29 & 14.34 & 0.01 & i & CPT 1m \\
2015-09-24 & 2457290.02 & 14.72 & 0.02 & V & COJ 1m \\
2015-09-24 & 2457290.02 & 14.21 & 0.02 & R & COJ 1m \\
2015-09-24 & 2457290.02 & 15.65 & 0.03 & B & COJ 1m \\
2015-09-24 & 2457290.15 & 17.55 & 0.09 & uvw1 & Swift \\
2015-09-24 & 2457290.15 & 16.30 & 0.06 & u & Swift \\
2015-09-24 & 2457290.15 & 15.66 & 0.05 & b & Swift \\
2015-09-24 & 2457290.15 & 18.66 & 0.12 & uvw2 & Swift \\
2015-09-24 & 2457290.15 & 14.77 & 0.04 & v & Swift \\
2015-09-24 & 2457290.16 & 19.00 & 0.12 & uvm2 & Swift \\
2015-09-25 & 2457290.88 & 14.65 & 0.08 & V & COJ 2m \\
2015-09-25 & 2457290.88 & 14.20 & 0.03 & R & COJ 2m \\
2015-09-25 & 2457290.88 & 13.90 & 0.07 & I & COJ 2m \\
2015-09-26 & 2457291.88 & 14.68 & 0.07 & V & COJ 2m \\
2015-09-26 & 2457291.88 & 14.19 & 0.03 & R & COJ 2m \\
2015-09-26 & 2457291.91 & 14.64 & 0.07 & V & COJ 2m \\
2015-09-26 & 2457291.91 & 14.68 & 0.07 & V & COJ 2m \\
2015-09-26 & 2457291.92 & 14.17 & 0.03 & R & COJ 2m \\
2015-09-26 & 2457291.92 & 13.86 & 0.06 & I & COJ 2m \\
2015-09-26 & 2457292.00 & 14.72 & 0.02 & V & COJ 1m \\
2015-09-26 & 2457292.00 & 14.24 & 0.02 & R & COJ 1m \\
2015-09-26 & 2457292.00 & 15.71 & 0.02 & B & COJ 1m \\
2015-09-26 & 2457292.02 & 14.66 & 0.07 & V & COJ 2m \\
2015-09-26 & 2457292.02 & 14.19 & 0.03 & R & COJ 2m \\
2015-09-26 & 2457292.03 & 13.86 & 0.06 & I & COJ 2m \\
2015-09-27 & 2457292.98 & 15.76 & 0.02 & B & COJ 1m \\
2015-09-27 & 2457292.99 & 15.76 & 0.02 & B & COJ 1m \\
2015-09-27 & 2457292.99 & 14.73 & 0.02 & V & COJ 1m \\
2015-09-27 & 2457292.99 & 14.73 & 0.02 & V & COJ 1m \\
2015-09-27 & 2457292.99 & 15.21 & 0.01 & g & COJ 1m \\
2015-09-27 & 2457293.00 & 15.23 & 0.01 & g & COJ 1m \\
2015-09-27 & 2457293.00 & 14.38 & 0.01 & r & COJ 1m \\
2015-09-27 & 2457293.00 & 14.37 & 0.01 & r & COJ 1m \\
2015-09-27 & 2457293.00 & 14.39 & 0.01 & i & COJ 1m \\
2015-09-27 & 2457293.00 & 14.42 & 0.01 & i & COJ 1m \\
2015-09-27 & 2457293.01 & 14.80 & 0.02 & V & COJ 1m \\
2015-09-27 & 2457293.01 & 14.21 & 0.02 & R & COJ 1m \\
2015-09-27 & 2457293.01 & 15.66 & 0.03 & B & COJ 1m \\
2015-09-28 & 2457293.71 & 17.82 & 0.10 & uvw1 & Swift \\
2015-09-28 & 2457293.71 & 16.65 & 0.07 & u & Swift \\
2015-09-28 & 2457293.71 & 15.92 & 0.05 & b & Swift \\
2015-09-28 & 2457293.72 & 18.93 & 0.13 & uvw2 & Swift \\
2015-09-28 & 2457293.72 & 14.91 & 0.05 & v & Swift \\
2015-09-28 & 2457293.72 & 19.32 & 0.14 & uvm2 & Swift \\
2015-09-28 & 2457293.94 & 14.71 & 0.07 & V & COJ 2m \\
2015-09-28 & 2457293.94 & 14.23 & 0.03 & R & COJ 2m \\
2015-09-28 & 2457293.94 & 13.94 & 0.07 & I & COJ 2m \\
2015-09-28 & 2457294.01 & 14.78 & 0.02 & V & COJ 1m \\
2015-09-28 & 2457294.01 & 14.26 & 0.02 & R & COJ 1m \\
2015-09-28 & 2457294.01 & 15.71 & 0.03 & B & COJ 1m \\
2015-09-28 & 2457294.02 & 14.74 & 0.07 & V & COJ 2m \\
2015-09-28 & 2457294.02 & 14.19 & 0.03 & R & COJ 2m \\
2015-09-28 & 2457294.02 & 13.95 & 0.06 & I & COJ 2m \\
2015-10-01 & 2457296.91 & 14.86 & 0.07 & V & COJ 2m \\
2015-10-01 & 2457296.92 & 14.28 & 0.03 & R & COJ 2m \\
2015-10-01 & 2457296.92 & 13.93 & 0.06 & I & COJ 2m \\
2015-10-01 & 2457297.00 & 15.95 & 0.02 & B & COJ 1m \\
2015-10-01 & 2457297.00 & 14.88 & 0.02 & V & COJ 1m \\
2015-10-01 & 2457297.00 & 14.86 & 0.07 & V & COJ 2m \\
2015-10-01 & 2457297.00 & 14.28 & 0.03 & R & COJ 2m \\
2015-10-01 & 2457297.00 & 14.30 & 0.02 & R & COJ 1m \\
2015-10-01 & 2457297.00 & 13.93 & 0.06 & I & COJ 2m \\
2015-10-01 & 2457297.00 & 14.03 & 0.02 & I & COJ 1m \\
2015-10-01 & 2457297.01 & 15.95 & 0.02 & B & COJ 1m \\
2015-10-01 & 2457297.01 & 14.88 & 0.02 & V & COJ 1m \\
2015-10-01 & 2457297.02 & 15.94 & 0.02 & B & COJ 1m \\
2015-10-01 & 2457297.02 & 14.34 & 0.02 & R & COJ 1m \\
2015-10-01 & 2457297.02 & 14.05 & 0.02 & I & COJ 1m \\
2015-10-01 & 2457297.02 & 15.95 & 0.02 & B & COJ 1m \\
2015-10-01 & 2457297.02 & 14.88 & 0.02 & V & COJ 1m \\
2015-10-01 & 2457297.02 & 14.89 & 0.02 & V & COJ 1m \\
2015-10-01 & 2457297.02 & 15.34 & 0.01 & g & COJ 1m \\
2015-10-01 & 2457297.03 & 15.35 & 0.01 & g & COJ 1m \\
2015-10-01 & 2457297.03 & 14.49 & 0.01 & r & COJ 1m \\
2015-10-01 & 2457297.03 & 14.48 & 0.01 & r & COJ 1m \\
2015-10-01 & 2457297.03 & 14.44 & 0.01 & i & COJ 1m \\
2015-10-01 & 2457297.03 & 14.43 & 0.01 & i & COJ 1m \\
2015-10-02 & 2457298.00 & 15.97 & 0.02 & B & COJ 1m \\
2015-10-02 & 2457298.00 & 14.86 & 0.02 & V & COJ 1m \\
2015-10-02 & 2457298.00 & 14.34 & 0.02 & R & COJ 1m \\
2015-10-02 & 2457298.01 & 14.04 & 0.02 & I & COJ 1m \\
2015-10-02 & 2457298.02 & 13.95 & 0.06 & I & COJ 2m \\
2015-10-02 & 2457298.03 & 17.94 & 0.11 & uvw1 & Swift \\
2015-10-02 & 2457298.04 & 16.98 & 0.08 & u & Swift \\
2015-10-02 & 2457298.04 & 16.09 & 0.05 & b & Swift \\
2015-10-02 & 2457298.04 & 19.12 & 0.16 & uvw2 & Swift \\
2015-10-02 & 2457298.07 & 15.06 & 0.06 & v & Swift \\
2015-10-02 & 2457298.07 & 19.62 & 0.28 & uvm2 & Swift \\
2015-10-03 & 2457299.00 & 14.87 & 0.07 & V & COJ 2m \\
2015-10-03 & 2457299.00 & 14.28 & 0.04 & R & COJ 2m \\
2015-10-03 & 2457299.00 & 15.97 & 0.02 & B & COJ 1m \\
2015-10-03 & 2457299.00 & 13.98 & 0.06 & I & COJ 2m \\
2015-10-03 & 2457299.01 & 14.91 & 0.02 & V & COJ 1m \\
2015-10-03 & 2457299.01 & 14.33 & 0.02 & R & COJ 1m \\
2015-10-03 & 2457299.01 & 14.04 & 0.02 & I & COJ 1m \\
2015-10-04 & 2457300.00 & 14.92 & 0.07 & V & COJ 2m \\
2015-10-04 & 2457300.00 & 14.29 & 0.03 & R & COJ 2m \\
2015-10-04 & 2457300.00 & 14.01 & 0.06 & I & COJ 2m \\
2015-10-04 & 2457300.00 & 14.31 & 0.08 & R & COJ 1m \\
2015-10-04 & 2457300.01 & 14.06 & 0.03 & I & COJ 1m \\
2015-10-05 & 2457301.00 & 15.07 & 0.03 & V & COJ 1m \\
2015-10-05 & 2457301.00 & 15.54 & 0.02 & g & COJ 1m \\
2015-10-05 & 2457301.00 & 15.58 & 0.02 & g & COJ 1m \\
2015-10-05 & 2457301.00 & 14.89 & 0.07 & V & COJ 2m \\
2015-10-05 & 2457301.00 & 14.30 & 0.03 & R & COJ 2m \\
2015-10-05 & 2457301.00 & 14.53 & 0.01 & r & COJ 1m \\
2015-10-05 & 2457301.00 & 13.98 & 0.06 & I & COJ 2m \\
2015-10-05 & 2457301.00 & 14.54 & 0.01 & r & COJ 1m \\
2015-10-05 & 2457301.01 & 14.44 & 0.02 & i & COJ 1m \\
2015-10-05 & 2457301.01 & 14.45 & 0.01 & i & COJ 1m \\
2015-10-06 & 2457302.00 & 14.90 & 0.02 & V & COJ 1m \\
2015-10-06 & 2457302.00 & 14.35 & 0.02 & R & COJ 1m \\
2015-10-06 & 2457302.01 & 13.92 & 0.02 & I & COJ 1m \\
2015-10-06 & 2457302.03 & 14.94 & 0.06 & V & COJ 2m \\
2015-10-06 & 2457302.03 & 14.34 & 0.03 & R & COJ 2m \\
2015-10-06 & 2457302.03 & 14.00 & 0.06 & I & COJ 2m \\
2015-10-07 & 2457303.00 & 14.96 & 0.03 & V & COJ 1m \\
2015-10-07 & 2457303.00 & 14.97 & 0.07 & V & COJ 2m \\
2015-10-07 & 2457303.00 & 14.37 & 0.03 & R & COJ 2m \\
2015-10-07 & 2457303.26 & 19.59 & 0.33 & uvw2 & Swift \\
2015-10-07 & 2457303.26 & 15.06 & 0.07 & v & Swift \\
2015-10-07 & 2457303.26 & 19.81 & 0.41 & uvm2 & Swift \\
2015-10-07 & 2457303.29 & 18.33 & 0.16 & uvw1 & Swift \\
2015-10-07 & 2457303.29 & 17.45 & 0.12 & u & Swift \\
2015-10-07 & 2457303.29 & 16.21 & 0.07 & b & Swift \\
2015-10-08 & 2457304.00 & 15.05 & 0.08 & V & COJ 2m \\
2015-10-08 & 2457304.00 & 14.39 & 0.03 & R & COJ 2m \\
2015-10-08 & 2457304.00 & 14.00 & 0.06 & I & COJ 2m \\
2015-10-08 & 2457304.01 & 14.11 & 0.05 & I & COJ 1m \\
2015-10-08 & 2457304.01 & 16.25 & 0.02 & B & COJ 1m \\
2015-10-08 & 2457304.01 & 15.04 & 0.02 & V & COJ 1m \\
2015-10-08 & 2457304.01 & 14.46 & 0.02 & R & COJ 1m \\
2015-10-08 & 2457304.01 & 14.10 & 0.02 & I & COJ 1m \\
2015-10-09 & 2457305.00 & 15.01 & 0.06 & V & COJ 2m \\
2015-10-09 & 2457305.00 & 16.19 & 0.02 & B & COJ 1m \\
2015-10-09 & 2457305.00 & 14.35 & 0.03 & R & COJ 2m \\
2015-10-09 & 2457305.00 & 14.03 & 0.06 & I & COJ 2m \\
2015-10-09 & 2457305.00 & 15.05 & 0.02 & V & COJ 1m \\
2015-10-09 & 2457305.01 & 14.43 & 0.02 & R & COJ 1m \\
2015-10-10 & 2457306.00 & 16.28 & 0.02 & B & COJ 1m \\
2015-10-10 & 2457306.00 & 15.07 & 0.02 & V & COJ 1m \\
2015-10-10 & 2457306.02 & 15.12 & 0.06 & V & COJ 2m \\
2015-10-10 & 2457306.02 & 14.42 & 0.03 & R & COJ 2m \\
2015-10-10 & 2457306.02 & 14.04 & 0.06 & I & COJ 2m \\
2015-10-11 & 2457307.00 & 15.08 & 0.07 & V & COJ 2m \\
2015-10-11 & 2457307.00 & 14.42 & 0.03 & R & COJ 2m \\
2015-10-11 & 2457307.00 & 14.05 & 0.06 & I & COJ 2m \\
2015-10-11 & 2457307.01 & 16.41 & 0.02 & B & COJ 1m \\
2015-10-11 & 2457307.02 & 14.00 & 0.02 & I & COJ 1m \\
2015-10-12 & 2457308.37 & 16.37 & 0.03 & B & CPT 1m \\
2015-10-12 & 2457308.37 & 15.08 & 0.02 & V & CPT 1m \\
2015-10-12 & 2457308.38 & 14.49 & 0.02 & R & CPT 1m \\
2015-10-13 & 2457308.89 & 15.13 & 0.07 & V & COJ 2m \\
2015-10-13 & 2457308.89 & 14.46 & 0.03 & R & COJ 2m \\
2015-10-13 & 2457308.89 & 14.10 & 0.06 & I & COJ 2m \\
2015-10-14 & 2457309.89 & 15.16 & 0.07 & V & COJ 2m \\
2015-10-14 & 2457309.89 & 14.49 & 0.03 & R & COJ 2m \\
2015-10-14 & 2457309.89 & 14.13 & 0.06 & I & COJ 2m \\
2015-10-14 & 2457309.99 & 16.51 & 0.02 & B & COJ 1m \\
2015-10-14 & 2457310.00 & 15.17 & 0.02 & V & COJ 1m \\
2015-10-14 & 2457310.00 & 15.16 & 0.02 & V & COJ 1m \\
2015-10-14 & 2457310.00 & 15.76 & 0.01 & g & COJ 1m \\
2015-10-14 & 2457310.01 & 14.69 & 0.01 & r & COJ 1m \\
2015-10-14 & 2457310.01 & 14.61 & 0.01 & i & COJ 1m \\
2015-10-14 & 2457310.02 & 15.17 & 0.07 & V & COJ 2m \\
2015-10-14 & 2457310.02 & 14.50 & 0.03 & R & COJ 2m \\
2015-10-14 & 2457310.02 & 14.08 & 0.08 & I & COJ 2m \\
2015-10-15 & 2457311.00 & 15.21 & 0.08 & V & COJ 2m \\
2015-10-15 & 2457311.01 & 14.46 & 0.04 & R & COJ 2m \\
2015-10-15 & 2457311.01 & 14.16 & 0.07 & I & COJ 2m \\
2015-10-16 & 2457311.98 & 16.48 & 0.02 & B & COJ 1m \\
2015-10-16 & 2457311.98 & 15.21 & 0.02 & V & COJ 1m \\
2015-10-16 & 2457311.99 & 14.52 & 0.02 & R & COJ 1m \\
2015-10-16 & 2457311.99 & 14.22 & 0.02 & I & COJ 1m \\
2015-10-16 & 2457312.00 & 15.18 & 0.06 & V & COJ 2m \\
2015-10-16 & 2457312.01 & 14.47 & 0.03 & R & COJ 2m \\
2015-10-16 & 2457312.01 & 14.08 & 0.06 & I & COJ 2m \\
2015-10-21 & 2457317.33 & 16.66 & 0.03 & B & CPT 1m \\
2015-10-21 & 2457317.33 & 16.62 & 0.03 & B & CPT 1m \\
2015-10-21 & 2457317.34 & 15.30 & 0.02 & V & CPT 1m \\
2015-10-21 & 2457317.34 & 15.30 & 0.02 & V & CPT 1m \\
2015-10-21 & 2457317.34 & 15.94 & 0.01 & g & CPT 1m \\
2015-10-21 & 2457317.34 & 15.93 & 0.01 & g & CPT 1m \\
2015-10-21 & 2457317.35 & 14.82 & 0.01 & r & CPT 1m \\
2015-10-21 & 2457317.35 & 14.80 & 0.01 & r & CPT 1m \\
2015-10-21 & 2457317.35 & 14.71 & 0.01 & i & CPT 1m \\
2015-10-21 & 2457317.35 & 14.74 & 0.01 & i & CPT 1m \\
2015-10-22 & 2457317.56 & 16.70 & 0.03 & B & LSC 1m \\
2015-10-22 & 2457317.56 & 16.72 & 0.03 & B & LSC 1m \\
2015-10-22 & 2457317.56 & 15.35 & 0.02 & V & LSC 1m \\
2015-10-22 & 2457317.57 & 15.34 & 0.02 & V & LSC 1m \\
2015-10-22 & 2457317.57 & 15.98 & 0.01 & g & LSC 1m \\
2015-10-22 & 2457317.57 & 15.99 & 0.01 & g & LSC 1m \\
2015-10-22 & 2457317.57 & 14.82 & 0.01 & r & LSC 1m \\
2015-10-22 & 2457317.57 & 14.81 & 0.01 & r & LSC 1m \\
2015-10-25 & 2457321.29 & 16.66 & 0.02 & B & CPT 1m \\
2015-10-25 & 2457321.29 & 16.65 & 0.02 & B & CPT 1m \\
2015-10-25 & 2457321.30 & 15.37 & 0.02 & V & CPT 1m \\
2015-10-25 & 2457321.30 & 15.37 & 0.02 & V & CPT 1m \\
2015-10-25 & 2457321.30 & 16.01 & 0.01 & g & CPT 1m \\
2015-10-25 & 2457321.30 & 15.99 & 0.01 & g & CPT 1m \\
2015-10-25 & 2457321.31 & 14.80 & 0.01 & r & CPT 1m \\
2015-10-25 & 2457321.31 & 14.82 & 0.01 & r & CPT 1m \\
2015-10-25 & 2457321.31 & 14.74 & 0.01 & i & CPT 1m \\
2015-10-26 & 2457321.55 & 16.65 & 0.10 & B & LSC 1m \\
2015-10-26 & 2457321.56 & 15.41 & 0.03 & V & LSC 1m \\
2015-10-26 & 2457321.56 & 15.35 & 0.03 & V & LSC 1m \\
2015-10-26 & 2457321.56 & 14.85 & 0.02 & r & LSC 1m \\
2015-10-26 & 2457321.57 & 14.74 & 0.03 & i & LSC 1m \\
2015-10-30 & 2457325.55 & 16.82 & 0.02 & B & LSC 1m \\
2015-10-30 & 2457325.55 & 16.80 & 0.02 & B & LSC 1m \\
2015-10-30 & 2457325.56 & 15.51 & 0.02 & V & LSC 1m \\
2015-10-30 & 2457325.56 & 15.50 & 0.02 & V & LSC 1m \\
2015-10-30 & 2457325.56 & 16.10 & 0.01 & g & LSC 1m \\
2015-10-30 & 2457325.56 & 16.10 & 0.01 & g & LSC 1m \\
2015-10-30 & 2457325.57 & 14.90 & 0.01 & r & LSC 1m \\
2015-10-30 & 2457325.57 & 14.88 & 0.01 & r & LSC 1m \\
2015-10-30 & 2457325.57 & 14.84 & 0.01 & i & LSC 1m \\
2015-10-30 & 2457325.57 & 14.84 & 0.01 & i & LSC 1m \\
2015-11-01 & 2457327.52 & 16.91 & 0.02 & B & LSC 1m \\
2015-11-01 & 2457327.52 & 16.89 & 0.02 & B & LSC 1m \\
2015-11-01 & 2457327.52 & 15.54 & 0.02 & V & LSC 1m \\
2015-11-01 & 2457327.52 & 15.53 & 0.02 & V & LSC 1m \\
2015-11-01 & 2457327.53 & 16.89 & 0.02 & g & LSC 1m \\
2015-11-01 & 2457327.53 & 14.93 & 0.01 & r & LSC 1m \\
2015-11-01 & 2457327.53 & 14.94 & 0.01 & r & LSC 1m \\
2015-11-01 & 2457327.54 & 14.89 & 0.01 & i & LSC 1m \\
2015-11-03 & 2457329.52 & 15.63 & 0.09 & v & Swift \\
2015-11-03 & 2457329.52 & 20.03 & 0.40 & uvm2 & Swift \\
2015-11-03 & 2457329.61 & 18.99 & 0.21 & uvw1 & Swift \\
2015-11-03 & 2457329.61 & 18.54 & 0.26 & u & Swift \\
2015-11-03 & 2457329.61 & 16.98 & 0.09 & b & Swift \\
2015-11-03 & 2457329.61 & 20.98 & 0.93 & uvw2 & Swift \\
2015-11-04 & 2457331.29 & 16.95 & 0.02 & B & CPT 1m \\
2015-11-04 & 2457331.30 & 16.93 & 0.03 & B & CPT 1m \\
2015-11-04 & 2457331.30 & 15.53 & 0.02 & V & CPT 1m \\
2015-11-04 & 2457331.30 & 15.59 & 0.02 & V & CPT 1m \\
2015-11-04 & 2457331.30 & 16.22 & 0.01 & g & CPT 1m \\
2015-11-04 & 2457331.31 & 16.28 & 0.01 & g & CPT 1m \\
2015-11-04 & 2457331.31 & 15.01 & 0.01 & r & CPT 1m \\
2015-11-04 & 2457331.31 & 15.01 & 0.01 & r & CPT 1m \\
2015-11-04 & 2457331.31 & 14.94 & 0.01 & i & CPT 1m \\
2015-11-04 & 2457331.31 & 14.97 & 0.01 & i & CPT 1m \\
2015-11-06 & 2457333.01 & 19.13 & 0.23 & uvw1 & Swift \\
2015-11-06 & 2457333.01 & 18.58 & 0.23 & u & Swift \\
2015-11-06 & 2457333.01 & 17.07 & 0.07 & b & Swift \\
2015-11-06 & 2457333.01 & 20.37 & 0.38 & uvw2 & Swift \\
2015-11-06 & 2457333.01 & 15.77 & 0.07 & v & Swift \\
2015-11-06 & 2457333.02 & 20.39 & 0.32 & uvm2 & Swift \\
2015-11-09 & 2457335.90 & 16.99 & 0.03 & B & COJ 1m \\
2015-11-09 & 2457335.90 & 16.99 & 0.03 & B & COJ 1m \\
2015-11-09 & 2457335.91 & 15.66 & 0.02 & V & COJ 1m \\
2015-11-09 & 2457335.91 & 15.67 & 0.02 & V & COJ 1m \\
2015-11-09 & 2457335.91 & 16.34 & 0.01 & g & COJ 1m \\
2015-11-09 & 2457335.91 & 16.35 & 0.01 & g & COJ 1m \\
2015-11-09 & 2457335.92 & 14.99 & 0.01 & r & COJ 1m \\
2015-11-09 & 2457335.92 & 15.00 & 0.01 & r & COJ 1m \\
2015-11-10 & 2457336.92 & 16.97 & 0.02 & B & COJ 1m \\
2015-11-10 & 2457336.92 & 17.02 & 0.03 & B & COJ 1m \\
2015-11-10 & 2457336.93 & 15.68 & 0.02 & V & COJ 1m \\
2015-11-10 & 2457336.93 & 15.64 & 0.02 & V & COJ 1m \\
2015-11-10 & 2457336.93 & 16.31 & 0.01 & g & COJ 1m \\
2015-11-10 & 2457336.93 & 16.32 & 0.01 & g & COJ 1m \\
2015-11-10 & 2457336.94 & 15.01 & 0.01 & r & COJ 1m \\
2015-11-10 & 2457336.94 & 15.04 & 0.01 & r & COJ 1m \\
2015-11-10 & 2457336.94 & 15.01 & 0.01 & i & COJ 1m \\
2015-11-10 & 2457336.94 & 15.02 & 0.01 & i & COJ 1m \\
2015-11-13 & 2457339.51 & 17.14 & 0.03 & B & LSC 1m \\
2015-11-13 & 2457339.52 & 17.12 & 0.03 & B & LSC 1m \\
2015-11-13 & 2457339.52 & 15.79 & 0.02 & V & LSC 1m \\
2015-11-13 & 2457339.52 & 15.73 & 0.02 & V & LSC 1m \\
2015-11-13 & 2457339.52 & 16.42 & 0.01 & g & LSC 1m \\
2015-11-13 & 2457339.52 & 16.37 & 0.01 & g & LSC 1m \\
2015-11-13 & 2457339.53 & 15.03 & 0.01 & r & LSC 1m \\
2015-11-13 & 2457339.53 & 15.05 & 0.01 & r & LSC 1m \\
2015-11-14 & 2457340.52 & 17.08 & 0.03 & B & LSC 1m \\
2015-11-14 & 2457340.53 & 15.78 & 0.02 & V & LSC 1m \\
2015-11-14 & 2457340.53 & 15.79 & 0.02 & V & LSC 1m \\
2015-11-17 & 2457344.27 & 16.49 & 0.01 & g & CPT 1m \\
2015-11-17 & 2457344.27 & 16.50 & 0.02 & g & CPT 1m \\
2015-11-17 & 2457344.27 & 17.19 & 0.03 & B & CPT 1m \\
2015-11-17 & 2457344.27 & 15.12 & 0.01 & r & CPT 1m \\
2015-11-17 & 2457344.27 & 17.20 & 0.04 & B & CPT 1m \\
2015-11-17 & 2457344.27 & 15.14 & 0.01 & r & CPT 1m \\
2015-11-17 & 2457344.28 & 15.16 & 0.01 & i & CPT 1m \\
2015-11-17 & 2457344.28 & 15.80 & 0.02 & V & CPT 1m \\
2015-11-17 & 2457344.28 & 15.16 & 0.01 & i & CPT 1m \\
2015-11-17 & 2457344.28 & 15.81 & 0.02 & V & CPT 1m \\
2015-11-18 & 2457345.04 & 19.58 & 0.35 & uvw1 & Swift \\
2015-11-18 & 2457345.04 & 20.10 & 0.95 & u & Swift \\
2015-11-18 & 2457345.04 & 17.24 & 0.08 & b & Swift \\
2015-11-18 & 2457345.04 & 19.95 & 0.29 & uvw2 & Swift \\
2015-11-18 & 2457345.05 & 16.03 & 0.07 & v & Swift \\
2015-11-18 & 2457345.05 & 20.26 & 0.30 & uvm2 & Swift \\
2015-11-22 & 2457348.91 & 17.27 & 0.06 & B & COJ 1m \\
2015-11-22 & 2457348.91 & 17.18 & 0.05 & B & COJ 1m \\
2016-02-27 & 2457446.27 & 19.68 & 0.20 & g & COJ 1m \\
2016-02-27 & 2457446.27 & 17.54 & 0.04 & r & COJ 1m \\
2016-02-27 & 2457446.27 & 18.22 & 0.14 & i & COJ 1m \\
2016-02-27 & 2457446.28 & 17.82 & 0.09 & i & COJ 1m \\
2016-02-27 & 2457446.28 & 18.56 & 0.13 & V & COJ 1m \\
2016-02-27 & 2457446.28 & 18.68 & 0.14 & V & COJ 1m \\
2016-02-29 & 2457447.62 & 19.51 & 0.21 & g & CPT 1m \\
2016-02-29 & 2457447.62 & 19.27 & 0.12 & g & CPT 1m \\
2016-02-29 & 2457447.62 & 17.64 & 0.03 & r & CPT 1m \\
2016-02-29 & 2457447.62 & 17.60 & 0.04 & r & CPT 1m \\
2016-02-29 & 2457447.62 & 17.88 & 0.06 & i & CPT 1m \\
2016-02-29 & 2457447.63 & 17.97 & 0.08 & i & CPT 1m \\
2016-02-29 & 2457447.63 & 18.91 & 0.15 & V & CPT 1m \\
2016-03-03 & 2457451.27 & 19.78 & 0.30 & B & COJ 1m \\
2016-03-03 & 2457451.27 & 18.82 & 0.10 & V & COJ 1m \\
2016-03-03 & 2457451.27 & 18.78 & 0.10 & V & COJ 1m \\
2016-03-04 & 2457451.61 & 17.70 & 0.07 & r & CPT 1m \\
2016-03-04 & 2457451.61 & 19.19 & 0.27 & V & CPT 1m \\
2016-03-04 & 2457451.62 & 18.05 & 0.14 & i & CPT 1m \\
2016-03-08 & 2457455.88 & 20.18 & 0.18 & B & LSC 1m \\
2016-03-08 & 2457455.88 & 20.13 & 0.18 & B & LSC 1m \\
2016-03-08 & 2457455.88 & 18.94 & 0.07 & V & LSC 1m \\
2016-03-08 & 2457455.89 & 18.93 & 0.07 & V & LSC 1m \\
2016-03-08 & 2457455.89 & 19.60 & 0.07 & g & LSC 1m \\
2016-03-08 & 2457455.89 & 19.42 & 0.06 & g & LSC 1m \\
2016-03-08 & 2457455.89 & 17.80 & 0.04 & r & LSC 1m \\
2016-03-08 & 2457455.90 & 17.78 & 0.04 & r & LSC 1m \\
2016-03-08 & 2457455.90 & 18.10 & 0.06 & i & LSC 1m \\
2016-03-08 & 2457455.90 & 18.20 & 0.07 & i & LSC 1m \\
2016-03-09 & 2457456.87 & 20.09 & 0.06 & B & LSC 1m \\
2016-03-09 & 2457456.88 & 20.25 & 0.06 & B & LSC 1m \\
2016-03-09 & 2457456.88 & 19.02 & 0.03 & V & LSC 1m \\
2016-03-09 & 2457456.88 & 18.94 & 0.03 & V & LSC 1m \\
2016-03-09 & 2457456.88 & 19.57 & 0.03 & g & LSC 1m \\
2016-03-09 & 2457456.89 & 19.64 & 0.03 & g & LSC 1m \\
2016-03-09 & 2457456.89 & 17.67 & 0.01 & r & LSC 1m \\
2016-03-09 & 2457456.89 & 17.67 & 0.01 & r & LSC 1m \\
2016-03-09 & 2457456.90 & 18.11 & 0.02 & i & LSC 1m \\
2016-03-21 & 2457468.86 & 20.24 & 0.11 & B & LSC 1m \\
2016-03-21 & 2457468.87 & 19.23 & 0.05 & V & LSC 1m \\
2016-03-21 & 2457468.87 & 19.29 & 0.05 & V & LSC 1m \\
2016-03-21 & 2457468.87 & 19.80 & 0.04 & g & LSC 1m \\
2016-03-21 & 2457468.88 & 19.83 & 0.05 & g & LSC 1m \\
2016-03-21 & 2457468.88 & 17.88 & 0.02 & r & LSC 1m \\
2016-03-21 & 2457468.88 & 17.88 & 0.02 & r & LSC 1m \\
2016-03-21 & 2457468.89 & 18.29 & 0.04 & i & LSC 1m \\
2016-03-21 & 2457468.89 & 18.33 & 0.03 & i & LSC 1m \\
2016-04-02 & 2457480.78 & 20.59 & 0.15 & B & LSC 1m \\
2016-04-02 & 2457480.79 & 20.38 & 0.12 & B & LSC 1m \\
2016-04-02 & 2457480.79 & 19.39 & 0.06 & V & LSC 1m \\
2016-04-02 & 2457480.79 & 19.37 & 0.06 & V & LSC 1m \\
2016-04-02 & 2457480.80 & 19.80 & 0.05 & g & LSC 1m \\
2016-04-02 & 2457480.80 & 19.90 & 0.06 & g & LSC 1m \\
2016-04-02 & 2457480.81 & 18.05 & 0.02 & r & LSC 1m \\
2016-04-02 & 2457480.81 & 18.11 & 0.02 & r & LSC 1m \\
2016-04-02 & 2457480.82 & 18.51 & 0.04 & i & LSC 1m \\
2016-04-02 & 2457480.82 & 18.51 & 0.04 & i & LSC 1m \\
2016-04-13 & 2457491.85 & 20.40 & 0.08 & B & LSC 1m \\
2016-04-13 & 2457491.86 & 20.76 & 0.08 & B & LSC 1m \\
2016-04-13 & 2457491.86 & 19.49 & 0.04 & V & LSC 1m \\
2016-04-13 & 2457491.87 & 19.47 & 0.05 & V & LSC 1m \\
2016-04-13 & 2457491.87 & 20.25 & 0.04 & g & LSC 1m \\
2016-04-13 & 2457491.88 & 20.18 & 0.04 & g & LSC 1m \\
2016-04-13 & 2457491.88 & 18.29 & 0.02 & r & LSC 1m \\
2016-04-13 & 2457491.88 & 18.28 & 0.02 & r & LSC 1m \\
2016-04-13 & 2457491.89 & 18.74 & 0.03 & i & LSC 1m \\
2016-04-13 & 2457491.89 & 18.67 & 0.03 & i & LSC 1m \\
2016-04-16 & 2457494.88 & 20.96 & 0.09 & B & LSC 1m \\
2016-04-16 & 2457494.88 & 20.84 & 0.08 & B & LSC 1m \\
2016-04-16 & 2457494.89 & 19.57 & 0.03 & V & LSC 1m \\
2016-04-16 & 2457494.89 & 19.68 & 0.04 & V & LSC 1m \\
2016-04-16 & 2457494.89 & 20.13 & 0.03 & g & LSC 1m \\
2016-04-16 & 2457494.90 & 20.16 & 0.03 & g & LSC 1m \\
2016-04-16 & 2457494.90 & 18.47 & 0.02 & r & LSC 1m \\
2016-04-16 & 2457494.91 & 18.49 & 0.02 & r & LSC 1m \\
2016-04-16 & 2457494.91 & 18.77 & 0.02 & i & LSC 1m \\
2016-04-16 & 2457494.92 & 18.81 & 0.03 & i & LSC 1m \\
2016-04-27 & 2457505.73 & 20.84 & 0.24 & B & LSC 1m \\
2016-04-27 & 2457505.74 & 20.70 & 0.18 & B & LSC 1m \\
2016-04-27 & 2457505.74 & 19.99 & 0.12 & V & LSC 1m \\
2016-04-27 & 2457505.75 & 19.90 & 0.09 & V & LSC 1m \\
2016-04-27 & 2457505.75 & 20.41 & 0.14 & g & LSC 1m \\
2016-04-27 & 2457505.76 & 20.51 & 0.15 & g & LSC 1m \\
2016-04-27 & 2457505.76 & 18.52 & 0.03 & r & LSC 1m \\
2016-04-27 & 2457505.76 & 18.43 & 0.04 & r & LSC 1m \\
2016-04-27 & 2457505.77 & 19.24 & 0.09 & i & LSC 1m \\
2016-04-27 & 2457505.77 & 19.02 & 0.09 & i & LSC 1m \\
2016-05-14 & 2457522.69 & 20.88 & 0.16 & B & LSC 1m \\
2016-05-14 & 2457522.69 & 20.14 & 0.16 & V & LSC 1m \\
2016-05-14 & 2457522.70 & 20.07 & 0.10 & V & LSC 1m \\
2016-05-14 & 2457522.70 & 20.33 & 0.08 & g & LSC 1m \\
2016-05-14 & 2457522.71 & 20.36 & 0.06 & g & LSC 1m \\
2016-05-14 & 2457522.71 & 18.64 & 0.03 & r & LSC 1m \\
2016-05-14 & 2457522.72 & 18.67 & 0.02 & r & LSC 1m \\
2016-05-14 & 2457522.72 & 19.27 & 0.05 & i & LSC 1m \\
2016-05-14 & 2457522.72 & 19.19 & 0.04 & i & LSC 1m \\
2016-06-09 & 2457548.82 & 21.60 & 0.19 & B & LSC 1m \\
2016-06-09 & 2457548.83 & 21.50 & 0.15 & B & LSC 1m \\
2016-06-09 & 2457548.83 & 20.48 & 0.08 & V & LSC 1m \\
2016-06-09 & 2457548.84 & 20.33 & 0.13 & V & LSC 1m \\
2016-06-10 & 2457549.86 & 21.75 & 0.15 & B & LSC 1m \\
2016-06-10 & 2457549.86 & 21.80 & 0.20 & B & LSC 1m \\
2016-06-10 & 2457549.87 & 20.38 & 0.08 & V & LSC 1m \\
2016-06-10 & 2457549.87 & 20.40 & 0.06 & V & LSC 1m \\
2016-06-10 & 2457549.88 & 20.71 & 0.05 & g & LSC 1m \\
2016-06-10 & 2457549.88 & 20.72 & 0.05 & g & LSC 1m \\
2016-06-10 & 2457549.89 & 18.99 & 0.02 & r & LSC 1m \\
2016-06-10 & 2457549.89 & 19.00 & 0.02 & r & LSC 1m \\
2016-06-10 & 2457549.89 & 19.40 & 0.05 & i & LSC 1m \\
2016-06-10 & 2457549.90 & 19.46 & 0.04 & i & LSC 1m \\
2016-07-06 & 2457575.78 & 21.03 & 0.16 & V & LSC 1m \\
2016-07-06 & 2457575.79 & 20.75 & 0.09 & V & LSC 1m \\
2016-07-06 & 2457575.79 & 21.20 & 0.08 & g & LSC 1m \\
2016-07-06 & 2457575.79 & 21.31 & 0.08 & g & LSC 1m \\
2016-07-06 & 2457575.80 & 19.49 & 0.02 & r & LSC 1m \\
2016-07-06 & 2457575.80 & 19.47 & 0.03 & r & LSC 1m \\
2016-07-06 & 2457575.81 & 19.90 & 0.07 & i & LSC 1m \\
2016-07-06 & 2457575.81 & 19.75 & 0.06 & i & LSC 1m \\
2016-08-01 & 2457601.79 & 20.93 & 0.21 & V & LSC 1m \\
2016-08-01 & 2457601.80 & 21.11 & 0.16 & V & LSC 1m \\
2016-08-01 & 2457601.80 & 21.58 & 0.12 & g & LSC 1m \\
2016-08-01 & 2457601.81 & 21.39 & 0.11 & g & LSC 1m \\
2016-08-01 & 2457601.81 & 19.74 & 0.04 & r & LSC 1m \\
2016-08-01 & 2457601.81 & 19.81 & 0.05 & r & LSC 1m \\
2016-08-01 & 2457601.82 & 19.85 & 0.08 & i & LSC 1m \\
2016-08-01 & 2457601.82 & 20.02 & 0.10 & i & LSC 1m \\
2016-08-27 & 2457628.29 & 21.69 & 0.21 & V & CPT 1m \\
2016-08-27 & 2457628.29 & 21.26 & 0.17 & V & CPT 1m \\
2016-08-27 & 2457628.30 & 21.36 & 0.08 & g & CPT 1m \\
2016-08-27 & 2457628.30 & 21.37 & 0.10 & g & CPT 1m \\
2016-08-27 & 2457628.31 & 20.33 & 0.06 & r & CPT 1m \\
2016-08-27 & 2457628.31 & 20.33 & 0.05 & r & CPT 1m \\
2016-08-27 & 2457628.31 & 20.47 & 0.07 & i & CPT 1m \\
2016-08-27 & 2457628.32 & 20.33 & 0.08 & i & CPT 1m \\
2016-09-23 & 2457655.04 & 22.31 & 0.34 & V & COJ 1m \\
2016-09-23 & 2457655.04 & 21.80 & 0.26 & V & COJ 1m \\
2016-09-23 & 2457655.05 & 22.15 & 0.18 & g & COJ 1m \\
2016-09-23 & 2457655.05 & 22.05 & 0.16 & g & COJ 1m \\
2016-09-23 & 2457655.06 & 20.45 & 0.06 & r & COJ 1m \\
2016-09-23 & 2457655.06 & 20.58 & 0.07 & r & COJ 1m \\
2016-09-23 & 2457655.06 & 20.86 & 0.12 & i & COJ 1m \\
2016-09-23 & 2457655.07 & 20.80 & 0.12 & i & COJ 1m \\
2016-09-26 & 2457658.03 & 22.28 & 0.51 & V & COJ 1m \\
2016-09-26 & 2457658.03 & 21.55 & 0.25 & V & COJ 1m \\
2016-09-26 & 2457658.04 & 21.64 & 0.19 & g & COJ 1m \\
2016-09-26 & 2457658.04 & 21.38 & 0.15 & g & COJ 1m \\
2016-09-26 & 2457658.05 & 20.54 & 0.11 & r & COJ 1m \\
2016-09-26 & 2457658.06 & 20.58 & 0.18 & i & COJ 1m \\
2016-09-26 & 2457658.06 & 20.51 & 0.17 & i & COJ 1m \\
2016-09-27 & 2457658.98 & 21.74 & 0.33 & V & COJ 1m \\
2016-09-27 & 2457658.99 & 20.44 & 0.12 & V & COJ 1m \\
2016-09-27 & 2457659.01 & 20.21 & 0.09 & r & COJ 1m \\
2016-09-29 & 2457661.36 & 21.43 & 0.32 & V & CPT 1m \\
2016-09-29 & 2457661.36 & 21.65 & 0.26 & V & CPT 1m \\
2016-09-29 & 2457661.37 & 22.32 & 0.24 & g & CPT 1m \\
2016-09-29 & 2457661.37 & 22.22 & 0.55 & g & CPT 1m \\
2016-09-29 & 2457661.38 & 20.75 & 0.15 & r & CPT 1m \\
2016-09-29 & 2457661.38 & 20.63 & 0.09 & r & CPT 1m \\
2016-09-29 & 2457661.39 & 20.92 & 0.23 & i & CPT 1m \\
2016-09-29 & 2457661.39 & 20.68 & 0.15 & i & CPT 1m \\
2016-10-01 & 2457663.37 & 20.94 & 0.21 & V & CPT 1m \\
2016-10-01 & 2457663.37 & 21.41 & 0.30 & V & CPT 1m \\
2016-10-02 & 2457663.62 & 22.15 & 0.25 & V & LSC 1m \\
2016-10-02 & 2457663.62 & 21.79 & 0.20 & V & LSC 1m \\
2016-10-02 & 2457663.63 & 22.22 & 0.19 & g & LSC 1m \\
2016-10-02 & 2457663.63 & 22.41 & 0.24 & g & LSC 1m \\
2016-10-02 & 2457663.64 & 20.74 & 0.10 & r & LSC 1m \\
2016-10-02 & 2457663.64 & 20.78 & 0.09 & r & LSC 1m \\
2016-10-02 & 2457663.65 & 21.37 & 0.29 & i & LSC 1m \\
2016-10-02 & 2457663.65 & 20.84 & 0.15 & i & LSC 1m \\
2017-09-17 & 2458014.40 & 21.77 & 1.06 & uvw2 & Swift \\
2017-09-17 & 2458014.41 & 21.37 & 0.98 & uvw2 & Swift \\
2017-09-20 & 2458017.16 & 21.39 & 0.67 & u & Swift \\
2017-09-20 & 2458017.17 & 21.50 & 1.00 & uvw2 & Swift \\\enddata 
\tablecomments{} 
\end{deluxetable}
 \label{tab:LcObs}
%-----------------------------------------
%%%%%%%%%%%%%%%%
%%%%%%%%%%%%%%%%%%%%%%%%%%%%%%%%%%%%%%%%%%%%%%%%%%
% Don't change these lines
\bsp	% typesetting comment
\label{lastpage}
\end{document}

% End of mnras_template.tex